% -*- mode: latex -*-

\mansection{setdiff}
\begin{mandesc}
  \short{setdiff}{set difference between two matrices, vectors or lists}
\end{mandesc}

% -- Calling sequence section
\begin{calling_sequence}
\begin{verbatim}
C = setdiff(A,B)
[C [,ind]] = setdiff(A,B, ind_type=str, which=str)
\end{verbatim}
\end{calling_sequence}
% -- Parameters
\begin{parameters}
  \begin{varlist}
    \vname{A, B}:  vectors or matrices of numbers (Mat or IMat), strings or cells , or lists.
    \vname{C}: matrix, vector or list with components (elements, rows or columns) of A which are not in B.
    \vname{ind}: vector of indices such that $C=A(kA)$
    \vname{ind_type = str}: named optional argument, a string among \verb+{"double","int"}+ (default is \verb+"double"+)
    which gives the type for the index vector  \verb+ind+.
    \vname{which = str}: named optional argument, a string among \verb+{"elements","columns","rows"}+ (default is
    \verb+"elements"+) or any non ambiguous abreviation. This option is useful only for matrices of floating 
       point numbers (Mat) or integer numbers (IMat) for which the setdiff operation can be done also for
       columns or rows.
  \end{varlist}
\end{parameters}

\begin{mandescription}
\begin{itemize}
\item By default  (or when which="elements" ) this function computes the set 
difference $C = A \setminus B$ between 2 vectors (of numbers, strings or cells) or between 2
lists considered as sets. $C$ contains the components of $A$ which are not in $B$. Note that 
multiple equal components in $A$ are reduced to one such component (if it is not in $B$) and that
$C$ is sorted in ascending order (only for vectors of numbers or strings).

If $A$ and/or $B$ are matrices they are considered as big vectors.
$C$ is output as a row vector if  A is a row vector, otherwise $C$ is a column vector.
  

\item  Using the option \verb+which = "rows"+ or \verb+which = "columns"+  the function 
  computes the set difference of the rows or of the columns of the two entry 
  matrices (feature available only for Mat and IMat matrices).

\end{itemize}


\end{mandescription}

\begin{examples}

\paragraph{example 1} with vectors of numbers:
\begin{mintednsp}{nsp}
A = [0.5, -1, 2, 1, 2, -1, 0.5, 2, -1, 2];
B = [2, 0.5];
C = setdiff(A,B)
[C, ind] = setdiff(A,B)
// we must have C = A(ind)
C.equal[A(ind)]

// same example but ind is output as a IMat "int32" vector
[C, ind] = setdiff(A,B, ind_type="int")
\end{mintednsp}

\paragraph{example 2} with vectors of strings:
\begin{mintednsp}{nsp}
A = ["toto", "foo", "bar", "toto", "foobar", "bar", "toto", "foo", "bar"]
B = [ "foo", "bar" ]
C = setdiff(A,B)
[C, ind] = setdiff(A,B)
// we must have C = A(ind)
C.equal[A(ind)]
\end{mintednsp}

\paragraph{example 3} with vectors of cells:
\begin{mintednsp}{nsp}
A = {"toto", [0,1;1,0], "foo", 1, 2, 1, "bar"}
B = { "foo", 2,  [0,1;1,0] }
C = setdiff(A,B)
[C, ind] = setdiff(A,B)
// we must have C = A(ind)
C.equal[A(ind)]
\end{mintednsp}

\paragraph{example 4} with lists:
\begin{mintednsp}{nsp}
A = list("toto", [0,1;1,0], "foo", 1, 2, 1, "bar")
B = list( "foo", 2,  [0,1;1,0] )
C = setdiff(A,B)
[C, ind] = setdiff(A,B)
// we must have C = A(ind)
C.equal[A.sublist[ind]]
\end{mintednsp}

\paragraph{example 5} setdiff of rows or columns:
\begin{mintednsp}{nsp}
// for columns
A = grand(2,6,"uin",-1,1)
B = grand(2,7,"uin",-1,1)
[C,ind] = setdiff(A,B,which="columns")  // or simply which="c"
// we must have C = A(:,ind)
C.equal[A(:,ind)]

// for rows
A = grand(8,2,"uin",-1,1)
B = grand(5,2,"uin",-1,1)
[C,ind] = union(A,B,which="rows",ind_type="int")  // or simply which="r"
// we must have C = A(ind,:)
C.equal[A(ind,:)]
\end{mintednsp}

\end{examples}

\begin{manseealso}
  \manlink{unique}{unique}, \manlink{union}{union}, \manlink{intersect}{intersect}, \manlink{setxor}{setxor}  
\end{manseealso}

% -- Authors
\begin{authors}
  Bruno Pincon
\end{authors}
