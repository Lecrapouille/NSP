% -*- mode: latex -*-

\mansection{has}
\begin{mandesc}
  \short{has}{(method) test if a vector or list has a given element}
\end{mandesc}

% -- Calling sequence section
\begin{calling_sequence}
\begin{verbatim}
b = M.has[elem]
[b, ind] = M.has[elem]
[b, i, j] = M.has[elem]
\end{verbatim}
\end{calling_sequence}
% -- Parameters
\begin{parameters}
  \begin{varlist}
    \vname{M}:  a vector or matrix (of numbers or strings) or a list
    or an array of cells
    \vname{elem}: any nsp object when M is a list or a cells array, a
    numerical scalar or vector or matrix when M is a numerical matrix 
    or a string, a vector/matrix of strings when M is a matrix
    of strings.
    \vname{b}: a scalar boolean (or a vector of booleans)               
    \vname{ind, i, j}: indices (or vectors of indices)
  \end{varlist}
\end{parameters}

\begin{mandescription}
\begin{itemize}
\item  If $M$ is a list or an array of cells, this method looks if $M$
  has the element $elem$. If yes $b$ is true and the first index $ind$ 
  of $elem$ in $M$ ($M(ind)==elem$) could be provided (for a array
  of cells you can use the third form to get the first index in terms
  of (row,column) indexing). When $elem$ is not in $M$ then $b$ is false
  and $ind$ or $i$ and $j$ are set to 0.

\item If $M$ is a numerical or strings matrix, $elem$ could be a scalar but
  also a vector or matrix. In this case $b$, $ind$ (or $i$
  and $j$ for (row,column) indexing) have the same meaning than before
  but take the same vector or matrix form than $elem$ ($b(i)$ is true
  if $M$ has $elem(i)$ as element.
\end{itemize}

\end{mandescription}

\begin{examples}
  
\paragraph{example 1} using has with a list or a cells array. 
\begin{mintednsp}{nspxxx}
L=list(0, "toto", [0;0], %t)
L.has["toto"]
L.has[%f]
[b,ind] = L.has["toto"]
// same example with an array of cells
C = {0, "toto", [0;0], %t}
C.has["toto"]
C.has[%f]
[b,ind] = C.has["toto"]
[b,i,j] = C.has["toto"]
\end{mintednsp}
   
\paragraph{example 2} using has with a numerical matrix
\begin{mintednsp}{nspxxx}
M=[0,-1, 2, 4, 9]
M.has[[2,3,4]]
[b,ind]=M.has[[2,3,4]]
\end{mintednsp}
   
\end{examples}

\begin{manseealso}
  \manlink{find}{find},\manlink{mfind}{mfind}, \manlink{bsearch}{bsearch}  
\end{manseealso}

% -- Authors
\begin{authors}
  Bruno Pincon
\end{authors}
