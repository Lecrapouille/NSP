% -*- mode: latex -*-
\mansection{linfn}
\begin{mandesc}
  \short{linfn}{infinity norm} \\ % 
\end{mandesc}
%\index{linfn}\label{linfn}
%-- Calling sequence section
\begin{calling_sequence}
\begin{verbatim}
  [x,freq]=linfn(G,PREC,RELTOL,options);  
\end{verbatim}
\end{calling_sequence}
%-- Parameters
\begin{parameters}
  \begin{varlist}
    \vname{G}: is a \verb!syslin! list
    \vname{PREC}: desired relative accuracy on the norm
    \vname{RELTOL}: relative threshold to decide when an eigenvalue can be  considered on the imaginary axis.
    \vname{options}: available options are \verb!'trace'! or \verb!'cond'!
    \vname{x} is the computed norm.
    \vname{freq}: vector
  \end{varlist}
\end{parameters}
\begin{mandescription}
  Computes the Linf (or Hinf) norm of \verb!G!
  This norm is well-defined as soon as the realization
  \verb!G=(A,B,C,D)! has no imaginary eigenvalue which is both 
  controllable and observable.\verb!freq! is a list of the frequencies for which \verb!||G||! is 
  attained,i.e., such that \verb!||G (j om)|| = ||G||!.
  If -1 is in the list, the norm is attained at infinity.
  If -2 is in the list, \verb!G! is all-pass in some direction so that 
  \verb!||G (j omega)|| = ||G||! for all frequencies omega.
  The algorithm follows the paper by G. Robel 
  (AC-34 pp. 882-884, 1989).
  The case \verb!D=0! is not treated separately due to superior 
  accuracy of the general method when \verb!(A,B,C)! is nearly 
  non minimal.
  The \verb!'trace'! option traces each bisection step, i.e., displays 
  the lower and upper bounds and the current test point.
  The \verb!'cond'! option estimates a confidence index on the computed 
  value and issues a warning if computations are 
  ill-conditioned
  In the general case (\verb!A! neither stable nor anti-stable), 
  no upper bound is  prespecified.
  If by contrast \verb!A! is stable or anti stable, lower
  and upper bounds are computed using the associated 
  Lyapunov solutions.
\end{mandescription}
%-- see also
\begin{manseealso}
  \manlink{h\_norm}{h-norm}  
\end{manseealso}
%-- Author
\begin{authors}
  P. Gahinet (INRIA)
\end{authors}
