% -*- mode: latex -*-
\mansection{evans}
\begin{mandesc}
  \short{evans}{Evans root locus}\\ % @mandesc@
\end{mandesc}
%-- Calling sequence section
\begin{calling_sequence}
\begin{verbatim}
  evans(n,d, [,kmax])
\end{verbatim}
\end{calling_sequence}
%-- Parameters
\begin{parameters}
  \begin{varlist}
    \vname{n,d}: Two polynomial.
    \vname{kmax}: maximum gain requested for the plot as a real scalar
  \end{varlist}
\end{parameters}

\begin{mandescription}
  The \verb!evans! function draws the Evans root locus for a linear system
  described by its transfer form \verb!(n(s),d(s))!. This is the locus of the
  roots of \verb!1+k*n(s)/d(s)!, in the complex plane. For a selected sample of
  gains \verb!k <= kmax!, the imaginary part of the roots of \verb!d(s)+k*n(s)!
  is plotted vs the real part. To obtain the gain at a given point of the locus
  you can simply execute the following instruction: \verb!z=[1,%i]*locate(1);!
  and \verb!k=-1/real(horner(n,z){1}/horner(d,z){1})!
  and click the desired point on the root locus. If the coordinates of the
  selected point are in the real \verb!2x1! vector \verb!P=locate(1)! this \verb!k!
  solves the equation \verb! k*n(w) + d(w) =0! with
  \verb!w=P(1)+%i*P(2)=[1,%i]*P!.
\end{mandescription}
%--example
\begin{examples}

  \noindent A first example
  \begin{mintednsp}{nsp}
    n = 352*poly(-5,'s'); d= poly([0,0,2000,200,25,1],'s','coeffs');
    evans(n,d,100);
    P=3.0548543 - 8.8491842*%i;    //P=selected point
    k=-1/real(horner(n,P){1}/horner(d,P){1});
    roots(d+k*n)     //contains P as particular root
  \end{mintednsp}

  \noindent A second example
  \begin{mintednsp}{nsp}
    s=poly(0,'s');n=1+s;
    d=real(poly([-1 -2 -%i %i],'s'));
    evans(n,d,100);
  \end{mintednsp}
  \noindent A third example
  \begin{mintednsp}{nsp}
    n=real(poly([0.1-%i 0.1+%i,-10],'s'));
    d=real(poly([-1 -2 -%i %i],'s'));
    evans(n,d,80);
  \end{mintednsp}
\end{examples}
%-- see also
\begin{manseealso}
  \manlink{kpure}{kpure} \manlink{krac2}{krac2} \manlink{locate}{locate}
\end{manseealso}
