% -*- mode: latex -*-
\mansection{bode}
\begin{mandesc}
  \short{bode}{Bode plot}\\
\end{mandesc}
%-- Calling sequence section
\begin{calling_sequence}
\begin{verbatim}
  bode(n,d,fmin=,fmax=,step=,frq=,title=,dom=)
\end{verbatim}
\end{calling_sequence}
%-- Parameters
\begin{parameters}
  \begin{varlist}
    \vname{n,d}: two polynomial matrices of size \verb!nx1!
    \vname{fmin}: a real for minimal frequency bounds (in Hz) or \verb!sym!
    \vname{fmax}: a real for maximal frequency bounds (in Hz).
    \vname{step}: a real (logarithmic step) or a string \verb!'auto'!.
    \vname{title}: a \verb!nx1! string matrix (captions for each curve).
    \vname{frq}: a row vector or a \verb!nx1! matrix giving frequencies (in Hz).
    \vname{dom}: time domain \verb!'c'! or \verb!'d'! or a positive real (default \verb!'c'!).
  \end{varlist}
\end{parameters}

\begin{mandescription}
  Bode plot, i.e magnitude and phase of the frequency response of the system
  \verb!n/d!. The system can be a continuous-time (\verb!dom='c'!) or
  discrete-time (\verb!dom='d'! or a positive real) SIMO system.

  The frequencies are given by the matrix \verb!frq! or computed from
  the bounds \verb!fmin! and \verb!fmax! (in Hz) and \verb!step! giving
  the ( logarithmic ) discretization step which is automatically adapted if
  (\verb!step='auto'!).
\end{mandescription}

%--example
\begin{examples}
  \begin{mintednsp}{nsp}
    s=poly(0,'s')
    n=(s^2+2*0.9*10*s+100);d=(s^2+2*0.3*10.1*s+102.01);
    n1=n*(s^2+2*0.1*15.1*s+228.01); d1=d*(s^2+2*0.9*15*s+225);
    bode([n;n1],[d;d1],fmin=0.01,fmax=100,title=['h1';'h'])
  \end{mintednsp}
\end{examples}

%-- see also
\begin{manseealso}
  \manlink{nyquist}{nyquist}
  \manlink{gainplot}{gainplot}
  %\manlink{black}{black}
  %\manlink{repfreq}{repfreq}
  %\manlink{g\_margin}{g-margin}
  %\manlink{p\_margin}{p-margin}
  %\manlink{calfrq}{calfrq}
  %\manlink{phasemag}{phasemag}
\end{manseealso}
