% -*- mode: latex -*-
\mansection{contour2d}
\begin{mandesc}
  \short{contour2d}{level curves of a surface on a 2 dimensional plot}\\
  \short{contourf}{level curves of a surface on a 2 dimensional plot}\\
\end{mandesc}
%-- Calling sequence section
\begin{calling_sequence}
\begin{verbatim}
  contour2d(x,y,z,nv,options=values)
  contourf(x,y,z,nv=,options=values)
\end{verbatim}
\end{calling_sequence}

%-- Parameters
\begin{parameters}
  \begin{varlist}
    \vname{x,y}: two real vectors of size \verb!n1! and \verb!n2! defining a grid.
    \vname{z}: real matrix of size \verb!(n1,n2)! giving the z-values of a surface on the grid
    \verb!(x,y)! or a nsp function !z=f(x,y)!.
    \vname{nv}: the requested level values or the number of contour levels.
    \begin{varlist}
      \vname{-}If \verb!nv! is scalar, its value gives the
      number of level curves equally spaced from zmin to zmax as
      follows \verb!z= zmin + (1:nv)*(zmax-zmin)/(nv+1)!
      Note that the \verb!zmin! and \verb!zmax!
      levels are not drawn (generically they are reduced to points)
      but they can be added with
      \begin{nspcode}
        [im,jm] = find(z == zmin); // or zmax
        plot2d(x(im)',y(jm)',-9)
      \end{nspcode}
      \vname{-}If \verb!nv! is a vector, \verb!nv(i)! gives
      the value of the ith level curve.
    \end{varlist}
    \vname{style}: optional argument which gives the color to use for each level.
    \vname{axesflag,frameflag}: see \manlink{plot2d}{plot2d}
    \vname{leg}: see \manlink{plot2d}{plot2d}
    \vname{leg_pos}: see \manlink{plot2d}{plot2d}
    \vname{logflag}: see \manlink{plot2d}{plot2d}
    \vname{nax}: see \manlink{plot2d}{plot2d}
    \vname{rect}: see \manlink{plot2d}{plot2d}
    \vname{strf}: deprecated replaced by specifying \verb!rect!, \verb!axesflag!.
    \vname{auto_axis}: see \manlink{plot2d}{plot2d}
    \vname{iso}: see \manlink{plot2d}{plot2d}
  \end{varlist}
\end{parameters}

\begin{mandescription}
  \verb!contour2d! draws level curves of a surface
  \verb!z=f(x,y)! on a 2D plot. The values of \verb!f(x,y)! are
  given by the matrix \verb!z! at the grid points defined by
  \verb!x! and \verb!y!. The levels to be displayed are given by \verb!nv!.

  \verb!contourf! paints between two consecutives level curves of the surface.

  The format used to display level curve values can be changed by
  using the command \verb!xset("fpf",string)! where \verb!string!
  gives the requested format in a C format syntax (for example
  \verb!string="%.3f"!). The format \verb!string=""! can be used
  to switch back to default format and \verb!string=" "! to suppress printing. This
  last feature is useful in conjunction with \manlink{legends}{legends} to display
  the level numbers in a legend and not directly onto the level curves as
  usual (see Examples).
\end{mandescription}

%--example
\begin{examples}

\noindent A first example

\begin{mintednsp}{nsp}
  contour2d(1:10,1:10,rand(10,10),5,rect=[0,0,11,11],style=1:5)
  // changing the format of the printing of the levels
  xset("fpf","%.2f");
  F=get_current_figure();
  F.invalidate[];
\end{mintednsp}

\noindent A second example

\begin{mintednsp}{nsp}
  xset('colormap',jetcolormap(11));
  x=linspace(0,2*%pi,20);y=linspace(1,10,50);
  levels=-10:2:10;
  xset('fpf',' ');
  contour2d(x,y,sin(x)'*y,levels,style=1:11)
  for i=1:11, plot2d(0,0,line_color=i,leg=string(levels(i)),leg_pos='urm');end
  xtitle("level curves of function sin(x)*y");
\end{mintednsp}

\noindent A third example with \verb!contourf!

\begin{mintednsp}{nsp}
  xset('colormap',jetcolormap(11));
  x=linspace(0,2*%pi,20);y=linspace(1,10,50);
  levels=-10:2:10;
  contourf(x,y,sin(x)'*y,nv=levels);
  xtitle("level curves of function sin(x)*y");
\end{mintednsp}

\end{examples}
%-- see also
\begin{manseealso}
  \manlink{contour}{contour} \manlink{contour2di}{contour2di} \manlink{plot2d}{plot2d}
\end{manseealso}
%-- Author
\begin{authors}
  J.-Ph. C.
\end{authors}
