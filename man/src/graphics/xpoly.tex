% -*- mode: latex -*-
\mansection{xpoly}
\begin{mandesc}
  \short{xpoly}{draw a polyline or a polygon}\\
\end{mandesc}
%-- Calling sequence section
\begin{calling_sequence}
\begin{verbatim}
  p=xpoly(x,y,options=value);
\end{verbatim}
\end{calling_sequence}
%-- Parameters
\begin{parameters}
  \begin{varlist}
    \vname{x,y}: two matrices of the same size (points of the polyline).
    \vname{close}: a boolean flag specifying to close or not the polyline.
    \vname{color}: color to be used for lines (default \verb!-1!).
    \vname{fill_color}: color for filling the polyline (default \verb!-2!).
    \vname{mark_color}: color ot the mark.
    \vname{mark_size}: size of font for mark.
    \vname{mark}: mark to be drawn at each point of the polyline (default \verb!-2!).
    \vname{thickness}: thickness of line.
    \vname{p}: a graphics polyline object.
  \end{varlist}
\end{parameters}
\begin{mandescription}
  \verb!xpoly! draws a single polyline described by the vectors of
  coordinates \verb!x! and \verb!y!. If \verb!x! and
  \verb!y! are matrices they are considered as vectors by
  concatenating their columns. The polyline is drawn using
  lines with color given by \verb!color!. If \verb!mark! is
  given the associated mark is also drawn at each polyline node.

  Negative values for some parameters have special meaning.
  Parameters are ignored when they are equal to \verb!-2!
  and are set to default values if they are equal to \verb!-1!.
\end{mandescription}
%--example
\begin{examples}

\noindent A poly mark

  \begin{mintednsp}{nsp}
    x=sin(2*%pi*(0:4)/5);
    y=cos(2*%pi*(0:4)/5);
    xsetech(frect=[-2,-2,2,2])
    xpoly(x,y,mark=4,mark_size=20,mark_color=6,close=%t,color=-2);
  \end{mintednsp}

\noindent A polyline

  \begin{mintednsp}{nsp}
    x=sin(2*%pi*(0:4)/5);
    y=cos(2*%pi*(0:4)/5);
    xsetech(frect=[-2,-2,2,2])
    xpoly(x,y,close=%f,color=3,thickness=4);
  \end{mintednsp}

\noindent polyline and poly mark

  \begin{mintednsp}{nsp}
    x=sin(2*%pi*(0:4)/5);
    y=cos(2*%pi*(0:4)/5);
    xsetech(frect=[-2,-2,2,2])
    xpoly(x,y,mark=4,mark_size=20,mark_color=6,close=%t,color=2,thickness=0);
  \end{mintednsp}

\end{examples}
%-- see also
\begin{manseealso}
  \manlink{xfpoly}{xfpoly} \manlink{xfpolys}{xfpolys} \manlink{xpolys}{xpolys}
\end{manseealso}
%-- Author
\begin{authors}
  J.-Ph. C.
\end{authors}
