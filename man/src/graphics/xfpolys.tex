% -*- mode: latex -*-

\mansection{xfpolys}
\begin{mandesc}
  \short{xfpolys}{fill a set of polygons}\\ % @mandesc@
\end{mandesc}
%-- Calling sequence section
\begin{calling_sequence}
\begin{verbatim}
  A=xfpolys(xpols,ypols,fill,options=value)
  A=xfpolys(xpols,ypols,options=value)
\end{verbatim}
\end{calling_sequence}

%-- Parameters
\begin{parameters}
  \begin{varlist}
    \vname{xpols,ypols}: two real matrices of size \verb!pxn!.
    \vname{fill}: vector of size \verb!n! or of size \verb!pxn!.
    \vname{A}: if requested \verb!A! is a graphic polyline object or a graphic compound object.
    \vname{color}: a scalar or a vector of size \verb!n!
    \vname{compound}: a boolean  specifying if returned value is a compound or the last polyline.
    \vname{fill_color}: a scalar or a vector of size \verb!n!
    \vname{thickness}:  a scalar or a vector of size \verb!n!
  \end{varlist}
\end{parameters}

\begin{mandescription}
  \verb!xfpolys! fills with colors a set of polygons of the same size defined by
  the two matrices \verb!xpols! and \verb!ypols!. The coordinates of the points of the
  \verb!i!-th polygon are given by columns \verb!i! of \verb!xpols! and \verb!ypols!.

  % The polygons may be filled with a given color (flat) or painted with
  % interpolated (shaded) colors.

  % \begin{description}
  % \item[flat color painting] In this case \verb!fill! should be a vector of size
  %   \verb!n!. The color to be used for filling polygon number \verb!i! is given by \verb!fill(i)!:
  %   \begin{description}
  %   \item[-] if \verb!fill(i)$<$0!, the polygon is filled with color \verb!-fill(i)!.
  %   \item[-] if \verb!fill(i)=0!, the polygon is drawn with the current color) and not filled.
  %   \item[-] if \verb!fill(i)$>$0!, the polygon is filled with color \verb!fill(i)! and
  %     the boundary is drawn with the current color.
  %   \end{description}
  % \item[interpolated color painting]
  %   In this case \verb!fill! should be a matrix with same sizes
  %   as \verb!xpols! and \verb!ypols!. Note that
  %   \verb!p! must be equal to 3 or 4.
  %   \verb!fill(k,i)! gives the color at the \verb!k!th edge
  %   of polygon \verb!i!.
  % \end{description}

  The argument \verb!fill! is used to specify the colors to be used for drawing and filling polygons
  The vector \verb!fill! should be a vector of size \verb!n! and the color to be used for filling
  polygon number \verb!i! is given by \verb!fill(i)!:
  \begin{description}
  \item[-] if \verb!fill(i)$<$0!, the polygon is filled with color \verb!-fill(i)!.
  \item[-] if \verb!fill(i)=0!, the polygon is drawn with the current color) and not filled.
  \item[-] if \verb!fill(i)$>$0!, the polygon is filled with color \verb!fill(i)! and
    the boundary is drawn with the current color.
  \end{description}

  The colors for drawing the boundaries and filling the polygons can also be given by
  optional named parameters. Negative values for some optional named parameters like \verb!color! or
  \verb!fill_color! have special meaning. Parameters are ignored when they are equal to \verb!-2!
  and are set to default values if they are equal to \verb!-1!.
\end{mandescription}

%--example
\begin{examples}

\noindent Using a \verb!fill! argument
\begin{mintednsp}{nsp}
  xsetech(frect=[-10,-10,210,40]);
  x1=[0,10,20,30,20,10,0]';
  y1=[15,30,30,15,0,0,15]';
  xpols=[x1 x1 x1 x1]; xpols=xpols+[0,60,120,180].*.ones(size(x1));
  ypols=[y1 y1 y1 y1];
  xset('thickness',3);
  xfpolys(xpols,ypols,[-1,0,10,13])
\end{mintednsp}

\noindent Using options

\begin{mintednsp}{nsp}
  xsetech(frect=[-10,-10,210,40]);
  x1=[0,10,20,30,20,10,0]';
  y1=[15,30,30,15,0,0,15]';
  xpols=[x1 x1 x1 x1]; xpols=xpols+[0,60,120,180].*.ones(size(x1));
  ypols=[y1 y1 y1 y1];
  xfpolys(xpols,ypols,color=[-2,-1,5,6], fill_color=[4,-2,10,13],thickness=3);
\end{mintednsp}

\noindent Mixing \verb!plot2d! and \verb!xfpolys!

\begin{mintednsp}{nsp}
function plot2d_polys(x,y,n,s,varargopt)
  plot2d(x,y,varargopt(:));
  z=ones(1,size(x,'*'));
  p=[x(:)';y(:)'];
  qx=p;qy=p;
  for i=1:n
    qx(i,:)=p(1,:)+s/2*cos(2*%pi*i/n);
    qy(i,:)=p(2,:)+s/2*sin(2*%pi*i/n);
  end
  v.fill_color= z* varargopt.find['line_color',def=-1];
  v.color=z;
  xfpolys(qx,qy,v(:));
endfunction

x=1:20;y=x.*sin(x);plot2d_polys(x,y/3,5,1.0,line_color=5)
\end{mintednsp}

\end{examples}

%-- see also
\begin{manseealso}
  \manlink{xfpoly}{xfpoly} \manlink{xpoly}{xpoly} \manlink{xpolys}{xpolys}
\end{manseealso}

%-- Author

\begin{authors}
  J.Ph.C.

\end{authors}
