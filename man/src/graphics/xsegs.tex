% -*- mode: latex -*-
\mansection{xsegs}
\begin{mandesc}
  \short{xsegs}{draw unconnected segments}\\
\end{mandesc}
%-- Calling sequence section
\begin{calling_sequence}
\begin{verbatim}
  xsegs(xv,yv,style=)
\end{verbatim}
\end{calling_sequence}

%-- Parameters
\begin{parameters}
  \begin{varlist}
    \vname{xv,yv}: matrices of the same size.
    \vname{style}: vector or scalar. If \verb!style! is a positive scalar,
    it gives the color to use for all segments. If \verb!style! is
    a negative scalar, then current color is used. If \verb!style!
    is a vector, then \verb!style(i)! gives the color to use for
    segment \verb!i!.
    \vname{thickness}: vector or scalar which gives the line thickness. 
    When a value is negative the default thickness is used.
    \end{varlist}
\end{parameters}

\begin{mandescription}
  \verb!xsegs! draws a set of unconnected segments given by \verb!xv! and
  \verb!yv!. If \verb!xv! and \verb!yv! are matrices they are considered as
  vectors by concatenating their columns. The coordinates of the two points
  defining a segment are given by two consecutive values of \verb!xv! and
  \verb!yv!:\verb!(xv(i),yv(i))--$>$(xv(i+1),yv(i+1))!. For instance, using
  matrices of size \verb!2xn!, the segments can be defined by:
  \begin{Verbatim}
    xv=[xi_1 xi_2 ...; xf_1 xf_2 ...] yv=[yi_1 yi_2 ...; yf_1 yf_2  ...]
  \end{Verbatim}
  and the segments are \verb!(xi_k,yi_k)--$>$(xf_k,yf_k)!.
\end{mandescription}
%--example
\begin{examples}
  \begin{mintednsp}{nsp}
    x=2*%pi*(0:9)/10;
    xv=[sin(x);9*sin(x)];
    yv=[cos(x);9*cos(x)];
    xsetech(frect=[-10,-10,10,10],fixed=%t);
    xsegs(xv,yv,style=1:10,thickness=2);
  \end{mintednsp}
\end{examples}
%-- Author
\begin{authors}
  J.-Ph. C.
\end{authors}
