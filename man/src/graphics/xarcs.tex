% -*- mode: latex -*-

\mansection{xarcs}
\begin{mandesc}
  \short{xarcs}{draw parts of a set of ellipses}\\ % @mandesc@
\end{mandesc}
%-- Calling sequence section
\begin{calling_sequence}
\begin{verbatim}
  R=xarcs(rects,[colors])
  R=xarcs(rects,color=..,thickness=..,background=..,compound=%t|%f);
\end{verbatim}
\end{calling_sequence}

%-- Parameters
\begin{parameters}
  \begin{varlist}
    \vname{arcs}: matrix of size (6,n) describing the ellipses.
    \vname{colors}: row vector of size n.
    \vname{background}: vector of size n.
    \vname{color}: vector of size n.
    \vname{thickness}: vector of size n.
    \vname{compound}: a boolean.
    \vname{R}: a graphic object.
  \end{varlist}
\end{parameters}

\begin{mandescription}
  \verb!xarcs! draws parts of a set of ellipses described by \verb!arcs!:
  \verb!arcs=[x y w h a1 a2;x y w h a1 a2;...]'! where each ellipse is defined
  by the 6 parameters \verb!(x,y,w,h,a1,a2)! (see \verb!xarc!).
  The value of \verb!colors(i)! gives the color to use for drawing rectangle \verb!i!.

  \verb!xarcs(arcs,color=..,thickness=..,background=..,compound=%t|%f);!
  draws parts of a set of ellipses. The color and thickness of drawing line can be given by optional arguments
  color and thickness. If optional background argument is given it is used as color to fill
  the rectangle. A graphic object is returned in argument \verb!R! if requested and by default
  its is a compound which contains all the graphics rectangle objects.
\end{mandescription}

%--example
\begin{examples}

\noindent A simple example

\begin{mintednsp}{nsp}
  xsetech(frect=[-1,-1,1,1]);
  arcs=[-1.0 0.0 0.5; // upper left x
  1.0 0.0 0.5; // upper left y
  0.5 1.0 0.5; // width
  0.5 0.5 1.0; // height
  0.0 0.0 0.0; // angle 1
  180*64 360*64 90*64]; // angle 2
  xarcs(arcs,color=[1,2,3],thickness=[3,3,3],background=[-2,5,6])
\end{mintednsp}

\noindent Random circles and ellipses

\begin{mintednsp}{nsp}
  n=100;x=rand(1,n);y=rand(1,n);w=0.1*rand(1,n);h=0.1*rand(1,n);
  arcs=[x;y;w;h;0*ones(1,n);360*64*ones(1,n)];
  xset('colormap',jetcolormap(32));
  xarcs(arcs,color=rand(1,n)*32,background=rand(1,n)*32);
\end{mintednsp}

\noindent Mixing plot2d and xarcs 

\begin{mintednsp}{nsp}
function plot2d_arcs(x,y,s,varargopt)
  plot2d(x,y,varargopt(:));
  z=ones(1,size(x,'*'));
  arcs=[x(:)'-s/2;y(:)'+s/2;s*z;s*z;0*z;360*64*z];
  v.background= z* varargopt.find['line_color',def=-1];
  v.color=z;
  v.thickness=z;
  xarcs(arcs,v(:));
endfunction

  x=1:20;y=x.*sin(x);plot2d_arcs(x,y/3,0.3,line_color=6);
\end{mintednsp}

\end{examples}

%-- see also
\begin{manseealso}
  \manlink{xarc}{xarc} \manlink{xfarc}{xfarc} \manlink{xfarcs}{xfarcs}
\end{manseealso}

%-- Author

\begin{authors}
  J.-Ph. C.;
\end{authors}
