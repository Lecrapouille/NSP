% -*- mode: latex -*-

\mansection{xrect}
\begin{mandesc}
  \short{xrect}{draw a rectangle}\\ % @mandesc@
\end{mandesc}
%-- Calling sequence section
\begin{calling_sequence}
\begin{verbatim}
  R=xrect(x,y,w,h,color=..,thickness=..,background=..)
  R=xrect(rect,color=..,thickness=..,background=..)
\end{verbatim}
\end{calling_sequence}

%-- Parameters
\begin{parameters}
  \begin{varlist}
    \vname{x,y,w,h}: four real values defining the rectangle.
    \vname{rect}: a vector of size \verb!4! giving the rectangle
    \verb![x,y,w,h]!.
    \vname{background, color, thickness}: optional arguments.
    \vname{R}: a graphical rectangle object.
  \end{varlist}
\end{parameters}

\begin{mandescription}
  \verb!xrect! draws a rectangle defined by \verb![x,y,w,h]!
  (upper-left point, width, height) using the current scale and style.
  The color and thickness of drawing line can be given by optional arguments
  \verb!color! and \verb!thickness!. If optional \verb!background!
  argument is given it is used as color to fill the rectangle.

  If \verb!color! is positive it gives the color to be used for drawing,
  if \verb!color=-1! the default color is used, if  \verb!color=-2! the
  rectangle is not drawn.

  If \verb!background! is positive it gives the color to be used for filling,
  if \verb!background=-1! the default filling color is used,
  if  \verb!background=-2! the rectangle is not filled.

\end{mandescription}

%--example
\begin{examples}

A first example

  \begin{mintednsp}{nsp}
    plot2d([-2,2],[-2,2],style=[-1,-1],strf="022");
    xrect(-1,1,2,2,color=6,thickness=4,background=3);
  \end{mintednsp}

A second example

  \begin{mintednsp}{nsp}
    plot2d([-2,2],[-2,2],style=[-1,-1],strf="022");
    xrect(-1,1,2,2,color=-1,thickness=4,background=3);
    // we change the default color
    xset('color',10);
    F=get_current_figure();
    F.invalidate[];// force redrawing;
  \end{mintednsp}
\end{examples}

%-- see also
\begin{manseealso}
  \manlink{xfrect}{xfrect} \manlink{xrects}{xrects}
\end{manseealso}

%-- Author
\begin{authors}
  J.Ph.C.
\end{authors}
