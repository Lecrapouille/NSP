% -*- mode: latex -*-

\mansection{is\_string\_in\_array}
\begin{mandesc}
 \shortunder{is\_string\_in\_array}{is_string_in_array}{check if a string is in an array (or is a valid abbreviation)}
\end{mandesc}
% -- Calling sequence section
\begin{calling_sequence}
\begin{verbatim}
 num = is_string_in_array(key, table, abbrev=bool, names=[arg_name,func_name])  
\end{verbatim}
\end{calling_sequence}
% -- Parameters
\begin{parameters}
  \begin{varlist}
    \vname{key}: string.
    \vname{table}: array of strings.
    \vname{abbrev=bool}: optional named argument, scalar boolean (default is \verb+%t+)
    \vname{names=[arg_name,func_name]}: optional named argumenta, a string matrix with 2 strings.
    \vname{rep}: position of \verb+key+ in \verb+table+ or non positive integer.
  \end{varlist}
\end{parameters}
\begin{mandescription}
  This function tests if a string \verb+key+ is among a given vector of strings \verb+table+
 or is an non ambiguous abbreviation of one of these strings (unless \verb+abbrev+ 
 is set to false in which case abbreviation are not considered). In case of success, 
 \verb+num+ is the index of \verb+key+ in  \verb+table+ otherwise, when the optional
 \verb+names+ argument is not provided, \verb+num+ is set to 0 
 (or to -1 when \verb+abbrev=%t+ and when \verb+key+ is an ambiguous abbreviation).
 
  This function could be useful to parse the value \verb+key+ of an input string argument 
 \verb+arg_name+ of a nsp function  \verb+func_name+. In this case of utilization, it is
 practical to provide the optional argument \verb+names=[arg_name,func_name]+ as an error 
 is automatically set with the message : ``Error:  argument \verb+arg_name+ of function  
\verb+func_name+  has a wrong value \verb+'key'+ expected values are ....".

\end{mandescription}
% --example 
\begin{examples}
\begin{mintednsp}{nsp}
function rep = color_number(color)
   table = ["black","red","blue","orange","yellow","green","brown"]
   rep = is_string_in_array(color,table,names=["color","color_number"])
endfunction
color_number("y") // should be 5
ok = execstr("color_number(""bl"")",errcatch=%t); // ambiguous abbreviation (an error should be set)
if ~ok then lasterror();end
color_number("blu") //should be 3
\end{mintednsp}
\end{examples}

% -- see also
\begin{manseealso}
\manlink{has}{has}
\end{manseealso}

\begin{authors}
  Jean-Philippe Chancelier, Bruno Pincon
\end{authors}


