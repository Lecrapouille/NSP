% -*- mode: latex -*-
%% Scilab ( http://www.scilab.org/ ) - This file is part of Scilab
%% Copyright (C) 1987-2016 - (INRIA)
%%
%% This program is free software; you can redistribute it and/or modify
%% it under the terms of the GNU General Public License as published by
%% the Free Software Foundation; either version 2 of the License, or
%% (at your option) any later version.
%%
%% This program is distributed in the hope that it will be useful,
%% but WITHOUT ANY WARRANTY; without even the implied warranty of
%% MERCHANTABILITY or FITNESS FOR A PARTICULAR PURPOSE.  See the
%% GNU General Public License for more details.
%%
%% You should have received a copy of the GNU General Public License
%% along with this program; if not, write to the Free Software
%% Foundation, Inc., 59 Temple Place, Suite 330, Boston, MA  02111-1307  USA
%%                                                                                

\mansection{group}
\begin{mandesc}
  \short{group}{group delay for digital filter} \\ % 
\end{mandesc}
%\index{group}\label{group}
%-- Calling sequence section
\begin{calling_sequence}
\begin{verbatim}
  [tg,fr]=group(npts,a1i,a2i,b1i,b2i)  
\end{verbatim}
\end{calling_sequence}
%-- Parameters
\begin{parameters}
  \begin{varlist}
    \vname{npts}: integer: number of points desired in calculation of group delay
    \vname{a1i}: in coefficient, polynomial, rational polynomial, or cascade polynomial form this variable is the transfer function of the filter. In coefficient polynomial form this is a vector of coefficients (see below).
    \vname{a2i}: in coeff poly form this is a vector of coeffs
    \vname{b1i}: in coeff poly form this is a vector of coeffs
    \vname{b2i}: in coeff poly form this is a vector of coeffs
    \vname{tg}: values of group delay evaluated on the grid fr
    \vname{fr}: grid of frequency values where group delay is evaluated
  \end{varlist}
\end{parameters}
\begin{mandescription}
  Calculate the group delay of a digital filter
  with transfer function h(z).
  The filter specification can be in coefficient form,
  polynomial form, rational polynomial form, cascade
  polynomial form, or in coefficient polynomial form.
  In the coefficient polynomial form the transfer function is
  formulated by the following expression\verb!h(z)=prod(a1i+a2i*z+z**2)/prod(b1i+b2i*z+z^2)!
\end{mandescription}
%--example 
\begin{examples}
  \begin{mintednsp}{nsp}
    z=poly(0,'z');
    h=z/(z-.5);
    [tg,fr]=group(100,h);
    plot2d(fr,tg)
  \end{mintednsp}
\end{examples}
%-- Author
\begin{authors}
    Carey Bunks  
\end{authors}
