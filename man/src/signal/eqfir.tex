% -*- mode: latex -*-
%% Scilab ( http://www.scilab.org/ ) - This file is part of Scilab
%% Copyright (C) 1987-2016 - (INRIA)
%%
%% This program is free software; you can redistribute it and/or modify
%% it under the terms of the GNU General Public License as published by
%% the Free Software Foundation; either version 2 of the License, or
%% (at your option) any later version.
%%
%% This program is distributed in the hope that it will be useful,
%% but WITHOUT ANY WARRANTY; without even the implied warranty of
%% MERCHANTABILITY or FITNESS FOR A PARTICULAR PURPOSE.  See the
%% GNU General Public License for more details.
%%
%% You should have received a copy of the GNU General Public License
%% along with this program; if not, write to the Free Software
%% Foundation, Inc., 59 Temple Place, Suite 330, Boston, MA  02111-1307  USA
%%                                                                                

\mansection{eqfir}
\begin{mandesc}
  \short{eqfir}{minimax approximation of FIR filter} \\ % 
\end{mandesc}
%\index{eqfir}\label{eqfir}
%-- Calling sequence section
\begin{calling_sequence}
\begin{verbatim}
  [hn]=eqfir(nf,bedge,des,wate)  
\end{verbatim}
\end{calling_sequence}
%-- Parameters
\begin{parameters}
  \begin{varlist}
    \vname{nf}: number of output filter points desired
    \vname{bedge}: Mx2 matrix giving a pair of edges for each band
    \vname{des}: M-vector giving desired magnitude for each band
    \vname{wate}: M-vector giving relative weight of error in each band
    \vname{hn}: output of linear-phase FIR filter coefficients
  \end{varlist}
\end{parameters}
\begin{mandescription}
  Minimax approximation of multi-band, linear phase, FIR filter
\end{mandescription}
%--example 
\begin{examples}
  \begin{mintednsp}{nsp}
    hn=eqfir(33,[0 .2;.25 .35;.4 .5],[0 1 0],[1 1 1]);
    [hm,fr]=frmag(hn,256);
    xbasc();plot2d(fr,hm,style=2),
  \end{mintednsp}
\end{examples}
%-- Author
\begin{authors}
    Carey Bunks  
\end{authors}
