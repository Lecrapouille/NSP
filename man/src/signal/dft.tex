% -*- mode: latex -*-
%% Scilab ( http://www.scilab.org/ ) - This file is part of Scilab
%% Copyright (C) 1987-2016 - (INRIA)
%%
%% This program is free software; you can redistribute it and/or modify
%% it under the terms of the GNU General Public License as published by
%% the Free Software Foundation; either version 2 of the License, or
%% (at your option) any later version.
%%
%% This program is distributed in the hope that it will be useful,
%% but WITHOUT ANY WARRANTY; without even the implied warranty of
%% MERCHANTABILITY or FITNESS FOR A PARTICULAR PURPOSE.  See the
%% GNU General Public License for more details.
%%
%% You should have received a copy of the GNU General Public License
%% along with this program; if not, write to the Free Software
%% Foundation, Inc., 59 Temple Place, Suite 330, Boston, MA  02111-1307  USA
%%                                                                                

\mansection{dft}
\begin{mandesc}
  \short{dft}{discrete Fourier transform} \\ % 
\end{mandesc}
%\index{dft}\label{dft}
%-- Calling sequence section
\begin{calling_sequence}
\begin{verbatim}
  [xf]=dft(x,flag);  
\end{verbatim}
\end{calling_sequence}
%-- Parameters
\begin{parameters}
  \begin{varlist}
    \vname{x}: input vector
    \vname{flag}: indicates dft (flag=-1)  or idft (flag=1)
    \vname{xf}: output vector
  \end{varlist}
\end{parameters}
\begin{mandescription}
  Function which computes dft of vector \verb!x!.
\end{mandescription}
%--example 
\begin{examples}
  \begin{mintednsp}{nsp}
    n=8;omega = exp(-2*%pi*%i/n);
    j=0:n-1;F=omega.^(j'*j);  //Fourier matrix
    x=1:8;x=x(:);
    F*x
    fft(x)
    dft(x,-1)
    inv(F)*x
    ifft(x)
    dft(x,1)
  \end{mintednsp}
\end{examples}
%-- see also
\begin{manseealso}
  \manlink{fft}{fft}  
\end{manseealso}
%-- Author
\begin{authors}
    Carey Bunks  
\end{authors}
