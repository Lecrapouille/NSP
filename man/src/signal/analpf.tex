% -*- mode: latex -*-
%% Scilab ( http://www.scilab.org/ ) - This file is part of Scilab
%% Copyright (C) 1987-2016 - (INRIA)
%%
%% This program is free software; you can redistribute it and/or modify
%% it under the terms of the GNU General Public License as published by
%% the Free Software Foundation; either version 2 of the License, or
%% (at your option) any later version.
%%
%% This program is distributed in the hope that it will be useful,
%% but WITHOUT ANY WARRANTY; without even the implied warranty of
%% MERCHANTABILITY or FITNESS FOR A PARTICULAR PURPOSE.  See the
%% GNU General Public License for more details.
%%
%% You should have received a copy of the GNU General Public License
%% along with this program; if not, write to the Free Software
%% Foundation, Inc., 59 Temple Place, Suite 330, Boston, MA  02111-1307  USA
%%                                                                                

\mansection{analpf}
\begin{mandesc}
  \short{analpf}{create analog low-pass filter} \\ % 
\end{mandesc}
% \index{analpf}\label{analpf}
% -- Calling sequence section
\begin{calling_sequence}
\begin{verbatim}
  [hs,pols,zers,gain]=analpf(n,fdesign,rp,omega)  
\end{verbatim}
\end{calling_sequence}
% -- Parameters
\begin{parameters}
  \begin{varlist}
    \vname{n}: positive integer: filter order
    \vname{fdesign}: string: filter design method: \verb!'butt'! or
    \verb!'cheb1'! or \verb!'cheb2'! or \verb!'ellip'!
    \vname{rp}: vector of size $2$ of error values for cheb1, cheb2 and ellip
    filters where only \verb!rp(1)! is used for cheb1 case, only \verb!rp(2)! is
    used for cheb2 case, and \verb!rp(1)! and \verb!rp(2)! are both used for
    ellip case. The elements of \verb!rp! should be in $]0,1[$.
    \begin{itemize}
    \item for cheb1 filters \verb!1-rp(1)<ripple<1! in passband
    \item for cheb2 filters \verb!0<ripple<rp(2)! in stopband
    \item for ellip filters \verb!1-rp(1)<ripple<1! in passband 
      \verb!0<ripple<rp(2)! in stopband
    \end{itemize}
    \vname{omega}: cut-off frequency of low-pass filter in Hertz
    \vname{hs}: rational polynomial transfer function
    \vname{pols}: poles of transfer function
    \vname{zers}: zeros of transfer function
    \vname{gain}: gain of transfer function
  \end{varlist}
\end{parameters}
\begin{mandescription}
  Creates analog low-pass filter with cut-off frequency at
  omega.\verb!hs=gain*poly(zers,'s')/poly(pols,'s')!
\end{mandescription}
%--example 
\begin{examples}
  \begin{mintednsp}{nsp}
    //Evaluate magnitude response of continuous-time system 
    hs=analpf(4,'cheb1',[.1 0],5);
    fr=0:.1:15;
    hf=freq(hs.num,hs.den,%i*fr);
    hm=abs(hf);
    plot2d(fr,hm)
  \end{mintednsp}
\end{examples}
%-- Author
\begin{authors}
  Carey Bunks  
\end{authors}
