% -*- mode: latex -*-
%% Scilab ( http://www.scilab.org/ ) - This file is part of Scilab
%% Copyright (C) 1987-2016 - (INRIA)
%%
%% This program is free software; you can redistribute it and/or modify
%% it under the terms of the GNU General Public License as published by
%% the Free Software Foundation; either version 2 of the License, or
%% (at your option) any later version.
%%
%% This program is distributed in the hope that it will be useful,
%% but WITHOUT ANY WARRANTY; without even the implied warranty of
%% MERCHANTABILITY or FITNESS FOR A PARTICULAR PURPOSE.  See the
%% GNU General Public License for more details.
%%
%% You should have received a copy of the GNU General Public License
%% along with this program; if not, write to the Free Software
%% Foundation, Inc., 59 Temple Place, Suite 330, Boston, MA  02111-1307  USA
%%                                                                                

\mansection{hilb}
\begin{mandesc}
  \short{hilb}{Hilbert transform} \\ % 
\end{mandesc}
%\index{hilb}\label{hilb}
%-- Calling sequence section
\begin{calling_sequence}
\begin{verbatim}
  [xh]=hilb(n[,wtype][,par])  
\end{verbatim}
\end{calling_sequence}
%-- Parameters
\begin{parameters}
  \begin{varlist}
    \vname{n}: odd integer: number of points in filter
    \vname{wtype}: string: window type \verb!('re','tr','hn','hm','kr','ch')! (default \verb!='re'!)
    \vname{par}: window parameter for \verb!wtype='kr' or 'ch'! default \verb!par=[0 0]! see the function window for more help
    \vname{xh}: Hilbert transform
  \end{varlist}
\end{parameters}
\begin{mandescription}
  returns the first n points of the
  Hilbert transform centred around the origin.
  That is, \verb!xh=(2/(n*pi))*(sin(n*pi/2))^2!.
\end{mandescription}
%--example 
\begin{examples}
  \begin{mintednsp}{nsp}
    plot2d([],hilb(51))
  \end{mintednsp}
\end{examples}
%-- Author
\begin{authors}
    Carey Bunks  
\end{authors}
