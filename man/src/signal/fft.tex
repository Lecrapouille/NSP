% -*- mode: latex -*-

\mansection{direct and inverse fast Fourier transform}
\begin{mandesc}
  \short{fft}{fast Fourier transform}\\
  \short{ifft}{inverse fast Fourier transform}\\
\end{mandesc}

% -- Calling sequence section
\begin{calling_sequence}
\begin{verbatim}
y = fft(x, dim=mode)
y = fft(x, mode)
y = ifft(x, dim=mode)
y = ifft(x, mode)
\end{verbatim}
\end{calling_sequence}
% -- Parameters
\begin{parameters}
  \begin{varlist}
    \vname{x}: a real or complex vector or matrix
    \vname{mode}: an integer or a string chosen among \verb+'M'+, \verb+'m'+, \verb+'full'+, \verb+'FULL'+, \verb+'row'+,
    \verb+'ROW'+, \verb+'col'+, \verb+'COL'+ or an non ambiguous abbreviation. 
    This argument is optional and if omitted 'full' is assumed.
    \vname{y}: resulting vector or matrix
  \end{varlist}
\end{parameters}

\begin{mandescription}

\verb+fft+ computes the discrete Fourier transform (dft) of a vector $x \in {\mathbb C}^m$:
$$
 y = F_m x, \quad  F_m = [ e^{-\frac{i 2 \pi k j}{m}} ]_{0 \le k, j \le m-1}
$$
and \verb+ifft+ computes the inverse transform, that is:
$$
 y = F_m^{-1} x, \quad  F_m^{-1} = \frac{1}{m} \bar{F}_m 
$$
using fast Fourier transform algorithms. In most cases nsp uses the routines from the fftw3 
library (if you compile nsp and fftw library is not found, fftpack is used instead). Note 
that fftw3 uses fast algorithms (in $m \log(m)$) for every $m$ even prime (but transforms 
are faster when $m$ is a power of 2).

The optional named argument \verb+dim+ have the following meaning:
\begin{itemize}
\item dim=0, (or "full", "FULL", "*") computes the fft or ifft of a row or column vector. If $x$ is a 
matrix, it is considered as a big column vector (using the column major order) on which the fft or 
ifft is computed but the result take the same form than $x$. This is the default.
\item dim=1, (or "row", "ROW") computes the fft or ifft of each column vectors of the matrix $x$ (column $j$
of $y$ correspond to the fft or ifft of column $j$ of $x$).
\item dim=2, (or "col", "COL") computes the fft or ifft of each row vectors of the matrix $x$ (row $i$
of $y$ correspond to the fft or ifft of row $i$ of $x$).
\item dim="m" (or "M") for Matlab compatibility: if $x$ is a vector computes its fft or ifft but if
$x$ is a matrix (with more than one row and one column) computes the fft or ifft of each column vectors 
of the matrix $x$.
\end{itemize}

\paragraph{Remarks}

\begin{enumerate}
\item One interpretation of the dft is ``approximation of some Fourier coefficients'' of
a $T$-periodic (complex valued) function $g$:
$$
      \hat{g}_k = ( g | \phi_k ) = \int_0^T g(t) \bar{\phi}_k(t) dt \mbox{ with } \phi_k(t) = \frac{1}{\sqrt{T}} e^{\frac{i 2\pi k t}{T}}
$$
by the rectangular rule using $m$ points uniformly spaced in $[0,T]$, $s_j = \Delta t \; j = (T/m) \; j, \quad j=0,\dots, m-1$:
$$
      \hat{g}_k \simeq c_k := \frac{T}{m} \sum_{j=0}^{m-1} g(s_j) \frac{1}{\sqrt{T}} e^{-\frac{i 2\pi  k s_j}{T}} 
             =  \frac{\sqrt{T}}{m} \sum_{j=0}^{m-1}  g(s_j)  e^{-\frac{2\pi i k j}{m}}
$$
This formula could give only $m$ different complex numbers because $c_{k+m} = c_k$ for every $k \in {\mathbb Z}$. So
it is possible to approximate Fourier coefficients only from $k = -(m-1)/2$ to  $k = (m-1)/2$ (when $m$ is
odd). Moreover negative modes are transposed as ``positive ones'' using the periodicity  $c_{k+m} = c_k$, and
$\hat{g}_{-k}$ is approximated by $c_{m-k}$, for $k=1,2, \dots$. This could be written:
$$
   c = \frac{\sqrt{T}}{m} F_m G, \mbox{  with } G = [ g(s_0), \dots, g(s_{m-1}) ]^{\top}
$$
This explains that the dft computes (approximated) Fourier coefficients not the usual order. The function 
\manlink{fftshift}{fftshift} can be used to recover it.

\item The dft corresponds also to (exact) trigonometric polynomial interpolation of a periodic function at regularly spaced 
points. If $g$ is such a $T$-periodic function, and one tries to interpolate it at the $s_j$ previously defined
by the trigonometric polynomial:
$$
   p(t) := \sum_{k=-n}^n c_k \phi_k(t), \quad n = (m-1)/2
$$
then the same formula to compute $c$ is obtained but always with the special order for the coefficients
(first corresponds to $c_0$, second to $c_1$, third is $c_2$, etc, last but one is $c_{-2}$ and 
last is $c_{-1}$).    

\end{enumerate}

\end{mandescription}
 
\begin{examples}
 \paragraph{simple example}
  \begin{mintednsp}{nsp}
// build a pure periodic signal from a trigonometric polynomial
// and recover its coefficients using fft
function x = poly_trigo(t, T, coef, num)
    x = zeros(size(t));
    for i=1:numel(coef), x = x + coef(i)*exp((2*%pi*%i*num(i)/T)*t)/sqrt(T); end
endfunction
// build a signal (real valued)
T = 2;   // period
coef = [0.5,   -1,  1, 0.86, 1,  -1, 0.5];  // coef
num  = [-12  , -4, -1, 0   , 1,   4,  12];  // mode number of the coef

// use a time discretisation and plot the signal
m = 101;
t = linspace(0, T, m+1)';
x = poly_trigo(t, T, coef, num);
xset("window",0)
xbasc()
plot2d(t, real(x), style=2)
xtitle("my signal")

// recover the coefficients using fft
c = (sqrt(T)/m)*real(fft(x(1:m)));
// reorder
c = fftshift(c);

xset("window",1)
xbasc()
// limit between mode -20 to 20 (modes other than -12, -4, -1, 0, 1, 4, 12
// should be null or numerically negligeable)
plot2d3((-20:20)',c(31:71),style=2, rect=[-21,-1.2,21,1.2])
plot2d3(num+0.2, coef, frameflag=0,style=5)
xtitle("blue: coefs from fft, red: exact coefs")

// in fact the coefs got from fft should be exact and this must be so
// if m >= 2*12+1 = 25. Try with m = 25
m = 25;
t = linspace(0, T, m+1)';
x = poly_trigo(t, T, coef, num);
c = (sqrt(T)/m)*real(fft(x(1:m)));
// reorder
c = fftshift(c);
xset("window",2)
xbasc()
plot2d3((-12:12)',c,style=2, rect=[-13,-1.2,13,1.2])
plot2d3(num+0.2, coef, frameflag=0,style=5)
xtitle("blue: coefs from fft, red: exact coefs")
  \end{mintednsp}
 \end{examples}

% \begin{examples}
%   \begin{itemize}
%     \item Comparison with explicit formula

%         a=[1;2;3];n=size(a,'*');
%         norm(1/n*exp(2*%i*%pi*(0:n-1)'.*.(0:n-1)/n)*a - ifft(a))
%         norm(exp(-2*%i*%pi*(0:n-1)'.*.(0:n-1)/n)*a - fft(a)) 

%     \item Frequency components of a signal, build a noise signal sampled at 1000Hz containing pure frequencies 
%       at 50 and 70 Hz

%         sample_rate=1000;
%         t = 0:1/sample_rate:0.6;
%         N=size(t,'*'); //number of samples
%         s=sin(2*%pi*50*t)+sin(2*%pi*70*t+%pi/4)+grand(1,N,'nor',0,1);
%         y=fft(s);
%         //the fft response is symetric we retain only the first N/2 points
%         f=sample_rate*(0:(N/2))/N; //associated frequency vector
%         n=size(f,'*')
%         plot2d(f,abs(y(1:n)))

%     \end{itemize}
% \end{examples}

\begin{manseealso}
  \manlink{fftshift}{fftshift}, \manlink{ifftshift}{ifftshift}
\end{manseealso}

% -- Authors
\begin{authors}
   fftw: M. Frigo,S.G. Johnson;  nsp interface: Bruno Pincon; fftpack: P.N. Swarztrauber
\end{authors}
