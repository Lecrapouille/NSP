% -*- mode: latex -*-
\mansection{findR}
\begin{mandesc}
  \short{findR}{Preprocessor for estimating the matrices of a linear time-invariant dynamical system} \\ % 
\end{mandesc}
%\index{findR}\label{findR}
%-- Calling sequence section
\begin{calling_sequence}
\begin{verbatim}
  [R,N [,SVAL,RCND]] = findR(S,Y,U,METH,ALG,JOBD,TOL,PRINTW)  
  [R,N] = findR(S,Y)  
\end{verbatim}
\end{calling_sequence}
%-- Parameters
\begin{parameters}
  \begin{varlist}
    \vname{S}: the number of block rows in the block-Hankel matrices.
    \vname{Y}:
    \vname{U}:
    \vname{METH}: an option for the method to use:
    \begin{varlist}
      \vname{1}:  MOESP method with past inputs and outputs;
      \vname{2}:  N4SI15     0     1     1  1000D method.
    \end{varlist}
    Default:    METH = 1.
    \vname{ALG}: an option for the algorithm to compute the triangular factor of the concatenated block-Hankel matrices built from the input-output data:
    \begin{varlist}
      \vname{1}:   Cholesky algorithm on the correlation matrix;
      \vname{2}:   fast QR algorithm;
      \vname{3}:   standard QR algorithm.
    \end{varlist}
    Default:    ALG = 1.
    \vname{JOBD}: an option to specify if the matrices B and D should later be computed using the MOESP approach:
    \begin{varlist}
      \vname{1}:  the matrices B and D should later be computed using the MOESP approach;
      \vname{2}:  the matrices B and D should not be computed using the MOESP approach.
    \end{varlist}
    Default: JOBD = 2. This parameter is not relevant for METH = 2.
    \vname{TOL}: a vector of length 2 containing tolerances:
    \begin{varlist}
      \vname{TOL(1)} is the tolerance for estimating the rank of matrices. If  TOL(1) $>$ 0,  the given value of  TOL(1)  is used as a lower bound for the reciprocal condition number.
      Default:    TOL(1) = prod(size(matrix))*epsilon\_machine where epsilon\_machine is the relative machine precision.
      \vname{TOL(2)} is the tolerance for estimating the system order. If  TOL(2) $>$= 0,  the estimate is indicated by the index of the last singular value greater than or equal to  TOL(2).  (Singular values less than  TOL(2) are considered as zero.)
      When  TOL(2) = 0,  then  S*epsilon\_machine*sval(1)  is used instead TOL(2),  where  sval(1)  is the maximal singular value. When  TOL(2) $<$ 0,  the estimate is indicated by the index of the singular value that has the largest logarithmic gap to its successor. Default:    TOL(2) = -1.
    \end{varlist}
    \vname{PRINTW}: a switch for printing the warning messages.
    \begin{varlist}
      \vname{1}: print warning messages;
      \vname{0}: do not print warning messages.
    \end{varlist}
    Default: PRINTW = 0.
    \vname{R}:
    \vname{N}: the order of the discrete-time realization
    \vname{SVAL}: singular values SVAL, used for estimating the order.
    \vname{RCND}: vector of length 2 containing the reciprocal condition numbers of the matrices involved in rank decisions or least squares solutions.
  \end{varlist}
\end{parameters}
\begin{mandescription}
  findR   Preprocesses the input-output data for estimating the matrices 
  of a linear time-invariant dynamical system, using Cholesky or
  (fast) QR factorization and subspace identification techniques 
  (MOESP or N4SID), and estimates the system order.
  [R,N] = findR(S,Y,U,METH,ALG,JOBD,TOL,PRINTW)  returns the processed
  upper triangular factor  R  of the concatenated block-Hankel matrices 
  built from the input-output data, and the order  N  of a discrete-time
  realization. The model structure is:
\begin{verbatim}
  x(k+1) = Ax(k) + Bu(k) + w(k),   k $>$= 1,
  y(k)   = Cx(k) + Du(k) + e(k).
\end{verbatim}
The vectors y(k) and u(k) are transposes of the k-th rows of Y and U,
respectively.
[R,N,SVAL,RCND] = findR(S,Y,U,METH,ALG,JOBD,TOL,PRINTW)  also returns
the singular values SVAL, used for estimating the order, as well as,
if meth = 2, the vector RCND of length 2 containing the reciprocal
condition numbers of the matrices involved in rank decisions or least
squares solutions.
[R,N] = findR(S,Y)  assumes U = [] and default values for the
remaining input arguments.
\end{mandescription}
%--example 
\begin{examples}
  \begin{mintednsp}{nsp}
    //generate data from a given linear system
    A = [ 0.5, 0.1,-0.1, 0.2;
      0.1, 0,  -0.1,-0.1;      
      -0.4,-0.6,-0.7,-0.1;  
      0.8, 0,  -0.6,-0.6];      
    B = [0.8;0.1;1;-1];
    C = [1 2 -1 0];
    SYS=syslin(0.1,A,B,C);
    U=ones(1,1000)+randn(1,1000);
    Y=flts(U,SYS)+0.5*randn(1,1000);
    // Compute R
    [R,N,SVAL] = findR(15,Y',U');
    SVAL
    N
  \end{mintednsp}
\end{examples}
%-- see also
\begin{manseealso}
  \manlink{findABCD}{findABCD} \manlink{findAC}{findAC} \manlink{findBD}{findBD} \manlink{findBDK}{findBDK} \manlink{sorder}{sorder} \manlink{sident}{sident}  
\end{manseealso}
