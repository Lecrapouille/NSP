% -*- mode: latex -*-

\mansection{int3d}
\begin{mandesc}
  \short{int3d}{three dimensional integration over a parallelepiped  or a set of tetrahedra}
\end{mandesc}

% -- Calling sequence section
\begin{calling_sequence}
\begin{verbatim}
Ia  = int3d(X, Y, Z, f)
[Ia, [,ea_estim, [,ier [,neval]]]] = int3d(X, Y, Z, f, atol=ea, rtol=er, args=fpar, limit=maxtet, vecteval=b)
\end{verbatim}
\end{calling_sequence}
% -- Parameters
\begin{parameters}
  \begin{varlist}
    \vname{X,Y,Z}: vectors with 2 components $X=[a,b]$, $Y=[c,d]$ and $Z = [e,f]$ for
    the parallelepiped case or  $4 \times n$ arrays given the coordinates of the $n$ tetrahedra.
    \vname{f}: nsp function
    \vname{atol=ea}: optional named argument, requested absolute error
                      on the result (default is 1e-12)
    \vname{rtol=er}: optional named argument, requested relative error
                      on the result (default is 1e-7).
    \vname{args=fpar}: optional named argument, useful to pass
    additional parameters needed by the function f.
    \vname{limit=maxtet}: optional named argument, max number of
    tetrahedra allowed in the subdivision process of the initial set of
    tetrahedra (default is 10000).
    \vname{vecteval=b}: optional named argument, boolean scalar
    (default \verb+%f+), use \verb+%t+ if you write the function
    \verb+f+ such that it could be evaluated on vectors arguments 
    (this speed up the computation).
    \vname{Ia}: computed approximate result
    \vname{ea_estim}: estimated absolute error
    \vname{ier}: scalar given some information about the computation.
    \vname{neval}: number of function evaluations used for the computation.
  \end{varlist}
\end{parameters}

\begin{mandescription}
This function computes an approximation $I_a$ of:
$$
   I = \int_a^b \int_c^d \int_e^f f(x,y,z) dxdydz
$$
or
$$
   I = \sum_k \int_{T_k} f(x,y,z) dxdydz
$$
such that:
$$
   | I - I_a | \le  \max \{ atol ,  rtol*|I| \}
$$
\verb+ea_estim+ is an estimation of $| I - I_a |$. The
\verb+ier+ variable could take the following values: 
\begin{itemize}
\item ier = 0,  normal and reliable termination of the routine. It is assumed that the
      requested  accuracy has been achieved.
\item ier = 1: maximum number of tetrahedra (the limit
      parameter) allowed has been achieved. You can try to
      increase limit.
\end{itemize}


Additional parameters for the function $f$ could be passed using the
named optional argument \verb+args=fpar+ where \verb+fpar+ is a cell array
with the additionnal parameters of $f$ : if $f$ is of the form 
\verb+function z=f(x,y,z,p1,p2,...,pn)+ then \verb+args={p1,p2,...,pn}+.
Note that if $f$ has only one additional parameter it is not mandatory 
to encapsulate it in a cell array (except when the parameter is it 
self a cell array). The additional parameters could be any nsp object.
 
The code uses the method described in ``Algorithm 720, TOMS ACM,
An Algorithm for Adaptive Cubature Over a Collection of 3-Dimensional Simplices
Jarle Berntsen, Ronald Cools, Terje O. Espelid''.

{\bf Rmk:} the computation should be much faster if you code the integrand function
such that it can be evaluated on vectors x, y and z (in this case you should add \verb+vecteval=%t+
to benefit of the speed up).

\end{mandescription}

\begin{examples}
  
\paragraph{example 1} integrating on the unit cube 
\begin{mintednsp}{nsp}
function v=f(x,y,z);v=cos(x+y+z);endfunction
[Ia,ea] = int3d([0,1],[0,1],[0,1],f)
I = -sin(3) + 3*(sin(2)-sin(1)) // exact result  
abs(Ia-I)
\end{mintednsp}
  
\paragraph{example 2} integrating on the unit simplex
\begin{mintednsp}{nsp}
X = [0;1;0;0];
Y = [0;0;1;0];
Z = [0;0;0;1];
function v=f(x,y,z);v=1-x-y-z;endfunction
[Ia,ea] = int3d(X,Y,Z,f)
I = 1/24 // exact result  
abs(Ia-I)
\end{mintednsp}
  
\paragraph{example 3} passing additionnal parameters to the integrand function.
A test function from the J. Berntsen, R. Cools, T. O. Espelid paper is the
following:
$$
 f(x,y,z) = \frac{1}{(1 + \alpha_1 x +  \alpha_2 y +  \alpha_3 z)^4 } 
$$
This function is integrated on the 3d unit simplex and exact result is (for $\alpha_i > -1$):
$$
I = \int_0^1 \int_0^{1-x} \int_0^{1-x-y} f(x,y,z) dz dy dx = \frac{1}{6 (1+\alpha_1)(1+\alpha_2)(1+\alpha_3)}
$$
\begin{mintednsp}{nsp}
function v=foo(x,y,z,alpha);
   v = 1./(1 + alpha(1)*x+alpha(2)*y+alpha(3)*z).^4;
endfunction
// the 3d unit simplex
X = [0;1;0;0];
Y = [0;0;1;0];
Z = [0;0;0;1];

// test 1
alpha = [3,13,9];
I = 1/(6*(1+alpha(1))*(1+alpha(2))*(1+alpha(3)))
// usual syntax 
[Ia,ea] = int3d(X,Y,Z,foo,args={alpha})
// but (only one additional parameter) you can use :
[Ia,ea] = int3d(X,Y,Z,foo,args=alpha)
abs(I - Ia)

// test2
alpha = [30,13,91];
I = 1/(6*(1+alpha(1))*(1+alpha(2))*(1+alpha(3)))
tic(); [Ia,ea] = int3d(X,Y,Z,foo,args=alpha); toc()
abs(I - Ia)

// test2 using vector evaluation (it should be much faster)
I = 1/(6*(1+alpha(1))*(1+alpha(2))*(1+alpha(3)))
tic(); [Ia,ea] = int3d(X,Y,Z,foo,args=alpha,vecteval=%t); toc()
abs(I - Ia)

// now split the vector of parameters in 3 scalar parameters
function v=foo(x,y,z,alpha1,alpha2,alpha3);
   v = 1./(1 + alpha1*x+alpha2*y+alpha3*z).^4;
endfunction
[Ia,ea] = int3d(X,Y,Z,foo,args={30,13,91},vecteval=%t)
abs(I - Ia)
\end{mintednsp}


\end{examples}

\begin{manseealso}
  \manlink{intg}{intg} ,  \manlink{int2d}{int2d}  
\end{manseealso}

% -- Authors
\begin{authors}
  Bruno Pincon
\end{authors}
