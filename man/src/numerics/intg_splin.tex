% -*- mode: latex -*-

\mansection{intg\_splin}
\begin{mandesc}
  \shortunder{intg\_splin}{intg_splin}{integrate a cubic spline on an interval}
\end{mandesc}

%-- Calling sequence section
\begin{calling_sequence}
    \begin{verbatim}
I = intg_splin(a, b, x, y, d)
I = intg_splin(a, b, x, y, d, outmode=str_outmode)
    \end{verbatim}
\end{calling_sequence}

%-- Parameters
\begin{parameters}
  \begin{varlist}
   \vname{a, b}: real scalars defining the integration interval
   \vname{x,y,d}: real vectors of same size defining a cubic spline or sub-spline function 
                (called $s$ in the following)
                \vname{str\_outmode}: (optional) string defining the evaluation of $s$ outside the $[x_1,x_n]$ interval
                (default is \verb+"by_nan"+)
   \vname{I}: result
  \end{varlist}
\end{parameters}

\begin{mandescription}
    Given three vectors $(x,y,d)$ defining a spline or sub-spline function
    (see \manlink{splin}{splin} ) with $y_i=s(x_i)$, $d_i = s'(x_i)$ this function
    computes:
$$
    I = \int_a^b s(t) dt
$$
    The \verb!outmode! parameter lets to define $s(t)$ for $t \notin [x_1,x_n]$.
 It can be choosen among \verb+"by_nan"+ (default),  \verb+"by_zero"+,  \verb+"C0"+,
    \verb+"linear"+,  \verb+"natural"+ or  \verb+"periodic"+. See \manlink{interp}{interp}
 help page for explanations.

  \end{mandescription}

  %--example 
  \begin{examples}

\begin{mintednsp}{nsp}
a = -8; b = 8;
function y=sinc(x),k=find(x==0),y=sin(x)./x,y(k)=1,endfunction
x = linspace(a,b,40)';
y = sinc(x);
dk = splin(x,y);

// integrate exactly the spline approximation of sinc
Ik = intg_splin(a,b,x,y,dk)

// integrate accuratly sinc 
I = intg(a,b,sinc)

// compute relative error
e = abs((Ik-I)/I)
\end{mintednsp}

\end{examples}

%-- see also
\begin{manseealso}
\manlink{splin}{splin}, \manlink{interp}{interp}, \manlink{intg}{intg}
\end{manseealso}

%-- Author
\begin{authors}
    Bruno Pincon
\end{authors}

