% -*- mode: latex -*-
\mansection{gamma,lngamma,digamma}
\begin{mandesc}
  \short{gamma}{Gamma function}\\
  \short{lngamma}{Logarithm of gamma function}\\
  \short{digamma}{logarithmic derivative of the gamma function}
\end{mandesc}
%-- Calling sequence section
\begin{calling_sequence}
\begin{verbatim}
  y = gamma(x);
  y = lngamma(x);  
  y = digamma(x);  
\end{verbatim}
\end{calling_sequence}
%-- Parameters
\begin{parameters}
  \begin{varlist}
    \vname{x,y}: real vector or matrix
  \end{varlist}
\end{parameters}

\begin{mandescription}
  The function \verb+gamma+ is defined by: 
  \[
  \operatorname(gamma) := \int_0^{+\infty} \exp(-y)y^{x-1} dy \,.
  \]
  The function \verb+lngamma+ is defined by:
  \[
  \operatorname{lngamma}(x) := ln(x)\Gamma(x) \,.
  \]
  The function \verb+digamma+ (also known as the \verb+psi+ function) 
  is defined by:
  \[
  \operatorname{digamma}(x) := Gamma'(x)/Gamma(x)\,.
  \]  
\end{mandescription}
%--example 
\begin{examples}
\begin{mintednsp}{nsp}
  function y=f(x); y = exp(-x.^2);endfunction;
  x=2;intg(0,x,f)*2/sqrt(%pi) - gamma(x);
\end{mintednsp}

\begin{mintednsp}{nsp}
function y=f(x); y = digamma(x);endfunction;
a=0.1;b=4;
y= abs(intg(a,b,f) - (lngamma(b) -lngamma(a)))
\end{mintednsp}

\end{examples}
%-- see also
%\begin{manseealso}
%\manlink{indexing arrays}{indexing arrays}
%\end{manseealso}

