% -*- mode: latex -*-
\mansection{rand}
\begin{mandesc}
  \short{rand}{generates random variates from the uniform distribution on [0,1)}
\end{mandesc}
%-- Calling sequence section
\begin{calling_sequence}
\begin{verbatim}
X = rand()     // output only one random number 
X = rand(m, n)
X = rand(Mat)
\end{verbatim}
\end{calling_sequence}
  %-- Parameters
\begin{parameters}
  \begin{varlist}
    \vname{m, n}: integers, size of the wanted matrix \verb!X!
   \vname{Mat}: a matrix whom only the dimensions (say \verb!m x n!) are used
   \vname{X}: the resulting \verb!m x n! random matrix
  \end{varlist}
  \end{parameters}
  
\begin{mandescription}
  This function generates random variates from the  \hyperlink{unfpdf}{uniform distribution} on $[0,1)$
 ($1$ is never returned). It is a shortcut for \verb+grand(m,n,"def")+ or
\verb+grand(m,n,"unf",0,1)+
  The first calling sequence \verb+rand()+ returns only one random number. The
 third calling sequence \verb+rand(Mat)+ is equivalent to the second  
 \verb+rand(m,n)+ with the dimensions $m$ and $n$ from the matrix $Mat$.

  The uniform distribution is got from the output of the basic generators which
 give random integers on $[0,n)$ (with $n = 2^{-32}$ in most cases) divided by $n$.
 
\end{mandescription}

%-- see also
%--example 
\begin{examples}
\begin{mintednsp}{nsp}
n = 1e5;
X = rand(n,1);  // a sample from U(0,1)

// draw an histogram
xbasc()
histplot(20,X,rect=[-0.2,0,1.2,1.2])
// superpose the uniform density
x = [-0.2,0,0,1,1,1.2];
y = [ 0  ,0,1,1,0,0];
xpoly(x,y,color=5,thickness=3)
\end{mintednsp}
\end{examples}

\begin{manseealso}
  \manlink{grand}{grand}, \manlink{randn}{randn}, \manlink{pdf}{pdf}   
\end{manseealso}



