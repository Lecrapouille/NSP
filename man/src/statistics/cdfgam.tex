% -*- mode: latex -*-
\mansection{cdfgam}
\begin{mandesc}
  \short{cdfgam}{cumulative distribution function gamma distribution}
\end{mandesc}
\index{cdfgam}\label{cdfgam}
  %-- Calling sequence section
\begin{calling_sequence}
\begin{verbatim}
[P,Q]=cdfgam("PQ",X,Shape,Rate)    // compute cdf
[X]=cdfgam("X",Shape,Rate,P,Q)     // compute inverse cdf (quantile function)
[Shape]=cdfgam("Shape",Rate,P,Q,X) // compute shape parameter 
[Rate]=cdfgam("Rate",P,Q,X,Shape)  // compute rate parameter
\end{verbatim}
\end{calling_sequence}
%-- Parameters
\begin{parameters}
  \begin{varlist}
    \vname{P,Q,X,Shape,Rate}: five real vectors of the same size.
    \vname{P,Q (Q=1-P)} the integral from 0 to X of the gamma density. Input range: [0,1] for inverse cdf and
          (0,1) for computing shape or rate parameter.
    \vname{X}: the upper limit of integration of the gamma density. Input range: [0, +infinity). Search range: [0,1E300]
      \vname{Shape}:  the shape parameter of the gamma density. Input range: (0, +infinity). Search range: [1E-300,1E300]
      \vname{Rate}:  the rate parameter of the gamma density. Input range: (0, +infinity). Search range: (1E-300,1E300].
 Note that usually the second parameter of the gamma distribution is a \verb!Scale! parameter equal to the inverse of the \verb!Rate! 
  (so you have to use \verb!Rate = 1/Shape!).
  \end{varlist}
\end{parameters}
\begin{mandescription}
  Calculates any one parameter of the gamma distribution given values for the others.
  The gamma density is given by for $x>0$ by~:
  \begin{equation}
    x^{\alpha-1} \frac{ \beta^\alpha  e^{- \beta x}}{\Gamma(\alpha)}
  \end{equation}
  where $\alpha = \mbox{\tt Shape}$ and $\beta = \mbox{\tt Rate}$. 
 


  Cumulative distribution function (P) is calculated directly by
  the code associated with: DiDinato, A. R. and Morris, A. H. 
  Computation of the  incomplete
  gamma function  ratios  and their  inverse.   ACM  Trans.  Math.
  Softw. 12 (1986), 377-393.
  Computation of other parameters involve a seach for a value that
  produces  the desired  value  of P.   The search relies  on  the
  monotinicity of P with the other parameter.
\end{mandescription}

\begin{examples}
\begin{mintednsp}{nsp}
shape =1;
rate =0.5;

// compute cdf
x=linspace(0,10,100)';
v = ones(size(x));
[p,q] = cdfgam("PQ",x,shape*v,rate*v); // you can alse use  cdf("gam",x,shape,scale)
xbasc()
plot2d(x,[p,q],style=[2,5],leg="cdf(x)@1-cdf(x)")

// compute inverse cdf (aka quantile)
xx = cdfgam("X", shape*v, rate*v, p, q); //  // you can use also icdf("gam",p,shape,scale,Q=q)

// relative error between x and xx 
er = max(abs(xx - x)./(x+1e-6))     // should be small


// compute shape parameter such that (p,q)=(0.2,0.8), x = 34 and rate = 5;
p = 0.2; q = 0.8; x = 34; rate = 5;
shape = cdfgam("Shape",rate,p,q,x)
// we should retrieve (p,q)=(0.2,0.8) ?
[pp,qq] = cdfgam("PQ",x,shape,rate)
abs(pp-p)/p    // should be small
abs(qq-q)/q    // should be small


// compute rate parameter such that (p,q)=(0.9,0.1), shape = 67, x = 2;
p = 0.9; q = 0.1; shape = 67; x = 2;  
rate = cdfgam("Rate",p,q,x,shape)
// we should retrieve (p,q)=(0.9,0.1) ?
[pp,qq] = cdfgam("PQ",x,shape,rate)
abs(pp-p)/p    // should be small
abs(qq-q)/q    // should be small

\end{mintednsp}
%function y=f(x)
%  y = x^(shape-1)*scale^shape*exp(-x*scale)/gamma(shape);
%endfunction
%
%z = zeros(size(x));
%for i=1:length(x)
%  z(i) = intg(0,x(i),f);
%end
%err=max(abs(y-z));

\end{examples}

\begin{authors}
  Nsp interface by Jean-Philippe Chancelier. Fortran code from DCDFLIB
  adapted in C and improved/modified by Bruno Pincon and Jean-Philippe Chancelier.  
  DCDFLIB: Library of Fortran Routines for Cumulative Distribution
  Functions, Inverses, and Other Parameters (February, 1994)
  Barry W. Brown, James Lovato and Kathy Russell. The University of Texas.

\end{authors}
