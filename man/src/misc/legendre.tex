% -*- mode: latex -*-

\mansection{legendre}
\begin{mandesc}
   \short{legendre}{associated Legendre functions}
\end{mandesc}

%-- Calling sequence section
\begin{calling_sequence}
\begin{verbatim}
 y = legendre(n,m,x [,normflag]) 
\end{verbatim}
\end{calling_sequence}

%-- Parameters
\begin{parameters}
  \begin{varlist}
   \vname{n}: non negative integer or vector of non negative integers regularly spaced with increment equal to 1
   \vname{m}: non negative integer or vector of non negative integers regularly spaced with increment equal to 1
   \vname{x}: real vector (elements of $x$ must be in $[-1,1]$)
   \vname{normflag}: (optional) scalar string
  \end{varlist}
\end{parameters}

\begin{mandescription}
  When $n$ and $m$ are scalars, \verb!legendre(n,m,x)! evaluates the associated Legendre 
  function $Pnm(x)$ at all the elements of $x$. The definition used is :
$$  
P_{nm}(x) = (-1)^m \left(1 - x^2 \right)^{m/2} \frac{d^m}{dx^m} P_n (x)
$$
where $Pn$ is the Legendre polynomial of degree $n$. So 
\verb!legendre(n,0,x)! evaluates the Legendre polynomial $Pn(x)$ at all 
the elements of $x$. 
  
When the normflag is equal to "norm" you get a normalized version (without
the \verb!(-1)^m! factor) :
$$
 P_{nm}^{norm}(x) = \sqrt{\frac{(2n+1) (n-m)!}{2 (n+m)!}} \left(1 - x^2 \right)^{m/2} \frac{d^m}{dx^m} P_n (x)
$$
which is useful to compute spherical harmonic functions (see Example 3).
  
For efficiency, one of the two first arguments may be a vector, for instance
\verb!legendre(n1:n2,0,x)! evaluates all the Legendre polynomials of
degree $n_1, n_1+1, ..., n_2$ at the elements of $x$ and
\verb!legendre(n,m1:m2,x)! evaluates all the Legendre associated 
functions $P_{nm}$ for $m=m_1, m_1+1, ..., m_2$ at $x$.

\paragraph{Output format}

In any case, the format of \verb!y! is \verb+max(length(n),length(m)) x length(x)+
with :
\begin{verbatim}
       y(i,j) = P(n(i),m;x(j))   if n is a vector
       y(i,j) = P(n,m(i);x(j))   if m is a vector
       y(1,j) = P(n,m;x(j))      if both n and m are scalars
\end{verbatim}
so that $x$ is preferably a row vector but any \texttt{mx x nx} matrix
is excepted and considered as an \texttt{1 x (mx * nx)} matrix, reshaped
following the column order.
  
\end{mandescription}

%--example 
\begin{examples}
\paragraph{example 1} plot the 6 first Legendre polynomials:
\begin{mintednsp}{nsp}
x = linspace(-1,1,200)';
y = legendre(0:5, 0, x);
xbasc()
plot2d(x,y', leg="p0@p1@p2@p3@p4@p5@p6")
xtitle("the 6 th first Legendre polynomials")
\end{mintednsp}

\paragraph{example 2} plot of the associated Legendre functions of degree 5 
\begin{mintednsp}{nsp}
x = linspace(-1,1,200)';
y = legendre(5, 0:5, x, "norm");
xbasc()
plot2d(x,y', leg="p5,0@p5,1@p5,2@p5,3@p5,4@p5,5")
xtitle("the (normalised) associated Legendre functions of degree 5")
\end{mintednsp}

\paragraph{example 3} plot spherical harmonics
\begin{mintednsp}{nsp}
// define Ylm functions
function [y] = harm_sph_Y(l,m,phi,theta)
   pl = legendre(l, abs(m), cos(phi), "norm")
   pl.redim[size(phi)]
   if m >= 0 then
      y = (-1)^m/(sqrt(2*%pi))*exp(%i*m*theta).*pl
   else
      y = 1/(sqrt(2*%pi))*exp(%i*m*theta).*pl
   end      
endfunction
// la suite un autre jour
\end{mintednsp}

\end{examples}

%-- Authors
\begin{authors}
Smith, John M. (code dxlegf.f from Slatec), B. Pincon (nsp interface + slight modifs on dxlegf.f) 
\end{authors}

