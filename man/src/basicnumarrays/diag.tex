\mansection{diag}
\begin{mandesc}
  \short{diag}{extract a diagonal of a matrix or build a diagonal matrix from a vector}
\end{mandesc}
%-- Calling sequence section
\begin{calling_sequence}
\begin{verbatim}
  // form 1
  A = diag(v)
  A = diag(v,k)

  // form 2
  v = diag(A)
  v = diag(A,k)
\end{verbatim}
\end{calling_sequence}
%-- Parameters
\begin{parameters}
  \begin{varlist}
    \vname{A}: matrix (full or sparse)
    \vname{v}: vector (full or sparse)
    \vname{k}: optional argument, must be an integer (default is $0$)
  \end{varlist}
\end{parameters}

\begin{mandescription}

  This function has two different behaviors depending if the first argument is a vector
  (call it form 1) or a matrix which is not a vector (called form 2).
  For both forms of this function, the optional argument $k$ have the same meaning and
 corresponds to the choosen diagonal number (see definition and a figure on the 
 \manlink{tril, triu}{tril} help page).

\begin{itemize}
\item \itemdesc{form 1}
   We get this behavior when the first argument, says \verb+v+ is detected as a vector, that is,
 it is a $m \times n$ matrix (full or sparse) with $m$ or $n$ equal to $1$. In this case the
 output  \verb+A+ is a matrix (full or sparse) with the $k$ th diagonal filled with the vector
 \verb+v+. If  \verb+v+ has $mn$ elements, we get a $p \times p$ (square) matrix (full or sparse)
 with $p=mn + | k |$. Note that it can be more efficient to use the \manlink{set_diag}{set_diag}
 method.
   
\item \itemdesc{form 2}
  To have this behavior, the first argument, says \verb+A+ should have been detected as 
a  $m \times n$ matrix (full or sparse) with both $m$ and $n$ differents of $1$. In this 
case the function extract the $k$ th diagonal of the matrix $A$ as a column vector (full or sparse).
\end{itemize}

\itemdesc{Remark:} When $v$ is a sparse vector, we get a sparse matrix $A$ and conversely.


\end{mandescription}

%--example 
\begin{examples}
\paragraph{first form examples}
\begin{mintednsp}{nsp}
v=rand(4,1) 
diag(v) 
diag(v,-1) 
diag(v,1)
// build the 1d discrete laplacien matrix (as a full matrix)
n = 5;
v = ones(n-1,1);
A = 2*eye(n) - diag(v,1) + diag(v,-1)
// build the 1d discrete laplacien matrix (as a sparse matrix)
n = 5;
v = sparse(ones(n-1,1));
A = 2*speye(n,n) - diag(v,1) + diag(v,-1)
\end{mintednsp}

\paragraph{second form examples}
\begin{mintednsp}{nsp}
A=rand(4,5) 
diag(A) 
diag(A,-1) 
diag(A,1)
\end{mintednsp}
\end{examples}

%-- see also
\begin{manseealso}
 \manlink{set_diag}{set_diag}, \manlink{eye}{eye}, \manlink{tril}{tril}, \manlink{triu}{triu}
\end{manseealso}

