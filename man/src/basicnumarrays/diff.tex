\mansection{diff}
\begin{mandesc}
  \short{diff}{difference of a vector or a matrix}
\end{mandesc}
%-- Calling sequence section
\begin{calling_sequence}
\begin{verbatim}
  d=diff(x)  
  d=diff(x,dim=dimval, order=orderval)  
  d=diff(x,dimval,orderval)
\end{verbatim}
\end{calling_sequence}
%-- Parameters
\begin{parameters}
  \begin{varlist}
    \vname{x}: numerical vector or matrix
    \vname{dimval}: A string chosen among \verb+'M'+, \verb+'m'+, \verb+'full'+, \verb+'FULL'+, \verb+'row'+,
    \verb+'ROW'+, \verb+'col'+, \verb+'COL'+ or an non ambiguous abbreviation or an integer. 
    This argument is optional and if omitted 'full' is assumed.
    \vname{orderval}: a non negative integer (default is 1)   
  \end{varlist}
\end{parameters}
\begin{mandescription}
  \verb+diff+ computes the difference of a vector \verb+x+ which is defined as follows~:
$$ d_1 = x_2 - x_1, \, d_2 = x_3 - x_2, \ldots, d_{n-1} = x_{n} - x_{n-1}\,.
$$
 Thus, the result is a vector (row if $x$ is row or column if $x$ is column) with 
 $n-1$ components if $x$ has $n$ components.

 Succesively applying the \verb+diff+ operation $k$ times can be performed by 
 using the optional \verb!order! parameter. For example, \verb+y=diff(x, order=2)+ gives 
 the same result as \verb+y=diff(diff(x))+ i.e a vector of length $n-2$ with following 
 components 
$$          
dd_1 = x_3 - 2x_2 + x_1, \, dd_2 = x_4 - 2x_3 + x_2, \ldots, dd_{n-2} = x_n  - 2x_{n-1} + x_{n-2} \,.
$$

The optional \verb!dim! argument can be given in order to compute the difference of the rows or columns vectors
of a matrix \verb+A+, precisely ($A$ being $m \times n$):
  \begin{description}
    \item['row' or 1]  computes the difference of the column vectors returning a $(m-order) \times n$ sized matrix.
    \item['col' or 2]  computes the difference of the rows vectors returning a $m \times (n-order)$ sized matrix.
    \item['full' or 0] (the default case) computes the difference of the matrix entries considered as 
           a big vector (using the column major order) returning a $(mn - order) \times 1$ sized matrix (or 
           a $1 \times (mn - order)$ sized matrix if the \verb+A+ matrix is a row vector).
    \item['m'] is the difference along the first non singleton dimension of the given matrix
          (for Matlab compatibility). 
  \end{description}
\end{mandescription}
%--example 
\begin{examples}
\paragraph{example 1} basic examples
\begin{mintednsp}{nsp}
x = 0:6;
diff(x)
x = (0:6).^2;
diff(x,order=1)
diff(x,order=2)
diff(x,order=3)
\end{mintednsp}

\paragraph{example 2} approximate derivatives of a function
\begin{mintednsp}{nsp}
n = 12;
x = linspace(0,2*%pi,n);
y = sin(x);
// diff(y)./diff(x) gives a good approximation
// of the derivatives at the middle points of the mesh.
// Here, since the mesh is uniform diff(x) could be replaced with
// the (uniform) step size 2*%pi/n
d = diff(y)/(2*%pi/n)
// the middle points:
xm = 0.5*(x(1:$-1) + x(2:$));
// the exact derivative
d_exact = cos(xm)
abs(d - d_exact)
\end{mintednsp}
\end{examples}

% -- see also
\begin{manseealso}
  \manlink{sum}{sum},\manlink{derivative}{derivative} 
\end{manseealso}

