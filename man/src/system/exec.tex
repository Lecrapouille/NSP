% -*- mode: latex -*-
\mansection{exec}
\begin{mandesc}
  \short{exec}{execute file contents} \\ % 
\end{mandesc}
%\index{exec}\label{exec}
%-- Calling sequence section
\begin{calling_sequence}
\begin{verbatim}
  [ok,renv]=exec(pathname, display=%t|%f, echo=%t|%f,
                 env=hashtable,
                 errcatch=%t| %f,
                 pausecatch=%t|%f)
  ok=exec(fname, display=%t|%f, echo=%t|%f,
           errcatch=%t| %f) 
  exec pathname
\end{verbatim}
\end{calling_sequence}
%-- Parameters
\begin{parameters}
  \begin{varlist}
    \vname{pathname}: a string giving the path of the file to be executed.
    \vname{fname}: a function name (not a primitive).
    \vname{display, echo, errcatch, pausecatch}: booleans optional values
    \vname{env, renv}: hash tables. 
    \vname{ok}: a boolean returned when \verb!errcatch! is et to \verb!%t!.
  \end{varlist}
\end{parameters}
\begin{mandescription}
  executes the contents of the file given by \verb!pathname! or execute 
  the body of a function given by \verb!fname!. 
  During execution of scripts the value of the current exec dir is 
  changed using the directory name of the given \verb!pathname!. 
  The current exec directory can be obtained with \verb!get_current_exec_dir! 
  which gives the current exec dir pathname relative to the current directory. 

  The evaluation can be controled by optional arguments. 
  
  \begin{itemize}
    \item \verb!display!: default value is false. When set to true the 
      display of evaluated expressions is done. 
    \item \verb!echo!: default value is false. When set to true the 
      command are echoed as the are parsed and evaluated. 
    \item \verb!errcatch!: default value is false. When set to true, 
      execution errors are catched and the status of the execution is 
      returned in the boolean variable \verb!ok!. In that case the 
      error message can be obtained with the \verb!lasterror! function.
    \item \verb!pausecatch!: default value is false. When set to true, 
      pause are not evaluated during the code execution. 
    \item \verb!env!: when given it gives an environment in which the 
      evaluation has to be done. In that case, the execution will 
      be performed in a private environment and the execution script 
      will not change the values stored in the current environement 
      (execept if values are returned throught the use of the function
      \verb!resume!).
    \item \verb!renv!: When \verb!renv! is requested the evaluation 
      is performed in a private environment which is enriched with 
      \verb!env! if \verb!env! is given and is returned by the function.
  \end{itemize}
\end{mandescription}
%--example 
% \begin{examples}
%   \begin{mintednsp}{nsp}
%   \end{mintednsp}
% \end{examples}
%-- see also
\begin{manseealso}
  \manlink{execstr}{execstr}
\end{manseealso}
