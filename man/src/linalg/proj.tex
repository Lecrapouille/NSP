% -*- mode: latex -*-
%% Scilab ( http://www.scilab.org/ ) - This file is part of Scilab
%% Copyright (C) 1987-2016 - F. Delebecque
%%
%% This program is free software; you can redistribute it and/or modify
%% it under the terms of the GNU General Public License as published by
%% the Free Software Foundation; either version 2 of the License, or
%% (at your option) any later version.
%%
%% This program is distributed in the hope that it will be useful,
%% but WITHOUT ANY WARRANTY; without even the implied warranty of
%% MERCHANTABILITY or FITNESS FOR A PARTICULAR PURPOSE.  See the
%% GNU General Public License for more details.
%%
%% You should have received a copy of the GNU General Public License
%% along with this program; if not, write to the Free Software
%% Foundation, Inc., 59 Temple Place, Suite 330, Boston, MA  02111-1307  USA
%%                                                                                                

\mansection{proj}
\begin{mandesc}
  \short{proj}{projection} \\ % 
\end{mandesc}
%\index{proj}\label{proj}
%-- Calling sequence section
\begin{calling_sequence}
\begin{verbatim}
  P = proj(X1,X2)    
\end{verbatim}
\end{calling_sequence}
%-- Parameters
\begin{parameters}
  \begin{varlist}
    \vname{X1,X2}: two real matrices with equal number of columns
    \vname{P}: real projection matrix (\verb!P^2=P!)
  \end{varlist}
\end{parameters}
\begin{mandescription}
  \verb!P! is the projection on \verb!X2! parallel to \verb!X1!.
\end{mandescription}
%--example 
\begin{examples}
  \begin{mintednsp}{nsp}
    X1=rand(5,2);X2=rand(5,3);
    P=proj(X1,X2);
    norm(P^2-P,1)
    trace(P)    // This is dim(X2)
    [Q,M]=fullrf(P);
    svd([Q,X2])   // span(Q) = span(X2)
  \end{mintednsp}
\end{examples}
%-- see also
\begin{manseealso}
  \manlink{projspec}{projspec} \manlink{orth}{orth} \manlink{fullrf}{fullrf}  
\end{manseealso}
%-- Author
\begin{authors}
  Fran�ois Delebecque
\end{authors}
