% -*- mode: latex -*-
%% Scilab ( http://www.scilab.org/ ) - This file is part of Scilab
%% Copyright (C) 1987-2016 - F. Delebecque
%%
%% This program is free software; you can redistribute it and/or modify
%% it under the terms of the GNU General Public License as published by
%% the Free Software Foundation; either version 2 of the License, or
%% (at your option) any later version.
%%
%% This program is distributed in the hope that it will be useful,
%% but WITHOUT ANY WARRANTY; without even the implied warranty of
%% MERCHANTABILITY or FITNESS FOR A PARTICULAR PURPOSE.  See the
%% GNU General Public License for more details.
%%
%% You should have received a copy of the GNU General Public License
%% along with this program; if not, write to the Free Software
%% Foundation, Inc., 59 Temple Place, Suite 330, Boston, MA  02111-1307  USA
%%
\mansection{arl2}
\begin{mandesc}
  \short{arl2}{SISO model realization by L2 transfer approximation} \\ % 
\end{mandesc}
% \index{arl2}\label{arl2}
% -- Calling sequence section
\begin{calling_sequence}
\begin{verbatim}
  [den,num,err]=arl2(y,den0,n [,imp=%f, all=%f])
  [h]=arl2(y,den0,n [,imp=%f, all=%f])
\end{verbatim}
\end{calling_sequence}
%-- Parameters
\begin{parameters}
  \begin{varlist}
    \vname{y}: real vector or polynomial in \verb!z^-1!, it contains the
    coefficients of the Fourier's series of the rational system to approximate
    (the impulse response) 
    \vname{den0}: a polynomial which gives an initial guess of the solution, it
    may be \verb!poly(1,'z','c')!
    \vname{n}: integer, the degree of approximating transfer function (degree of den)
    \vname{imp}: optional argument in \verb!(0,1,2)! (verbose mode)
    \vname{den}: polynomial or vector of polynomials, contains the
    denominator(s) of the solution(s)
    \vname{num}: polynomial or vector of polynomials, contains the numerator(s)
    of the solution(s)
    \vname{h}: \verb!num ./ den!
    \vname{err}: real constant or vector giving the l2-error achieved for each solutions
  \end{varlist}
\end{parameters}
\begin{mandescription}
  \verb![den,num,err]=arl2(y,den0,n [,imp]) ! finds a pair of polynomials
  \verb!num! and \verb!den! such that the transfer function \verb!num/den!
  is stable and its impulse response approximates (with a minimal l2
  norm) the vector \verb!y! assumed to be completed by an infinite number of zeros.
  If \verb!y(z)  =  y(1)(1/z)+y(2)(1/z^2)+ ...+ y(ny)(1/z^ny)!
  then l2-norm of \verb!num/den - y(z)! is \verb!err!. The integer 
  \verb!n! is the degree of the polynomial \verb!den!.
  The \verb!num/den!  transfer function is a L2 approximant of the
  Fourier's series of the rational system.
  Various intermediate results are printed according to optional \verb!imp! argument. 
  When the optional argument \verb!all! is set to true \verb!arl2! returns in the
  vectors of polynomials \verb!num! and \verb!den!  a set of local
  optimums for the problem. The solutions are sorted with increasing
  errors \verb!err!. In this last case \verb!den0! is already assumed to be 
  \verb!poly(1,'z','c')!.
\end{mandescription}
%--example 
\begin{examples}
  \begin{mintednsp}{nsp}
    v=ones(1,20);
    xbasc();
    plot2d1([],[v';zeros(80,1)],style=2,rect=[1,-0.5,100,1.5]);
    [d,n,e]=arl2(v,poly(1,'z','c'),1);
    plot2d1([],ldivp(n,d,100),style=2);
    [d,n,e]=arl2(v,d,3);
    plot2d1([],ldivp(n,d,100),style=3);
    [d,n,e]=arl2(v,d,8);
    plot2d1([],ldivp(n,d,100),style=5);
    [d,n,e]=arl2(v,poly(1,'z','c'),4,all=%t);
    plot2d1([],ldivp(n(1),d(1),100),style=10);
  \end{mintednsp}
\end{examples}
%-- see also
\begin{manseealso}
  \manlink{ldiv}{ldiv}
\end{manseealso}
