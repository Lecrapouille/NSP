% -*- mode: latex -*-
% Copyright (C) CeCILL Inria (scilab)
% -*- mode: latex -*-
\mansection{repfreq}
\begin{mandesc}
  \short{repfreq}{frequency response} \\ % 
\end{mandesc}
%\index{repfreq}\label{repfreq}
%-- Calling sequence section
\begin{calling_sequence}
\begin{verbatim}
 [frq,repf,split]= repfreq(n,d,fmin=,fmax=,step=,frq=)   
 [repf]=repfreq(n,d,fmin=,fmax=,step=,frq=)  
\end{verbatim}
\end{calling_sequence}
%-- Parameters
\begin{parameters}
  \begin{varlist}
    \vname{n,d}: two polynomial matrices of size \verb!nx1!
    \vname{fmin}: a real for minimal frequency bounds (in Hz) or \verb!sym!
    \vname{fmax}: a real for maximal frequency bounds (in Hz).
    \vname{step}: a real (logarithmic step) or a string \verb!'auto'!.
    \vname{frq}: a row vector or a \verb!nx1! matrix giving frequencies (in Hz).
    \vname{dom}: time domain \verb!'c'! or \verb!'d'! or a positive real (default \verb!'c'!).
    \vname{split}: vector of indexes of critical frequencies.
  \end{varlist}
\end{parameters}
\begin{mandescription}
  \verb!repfreq! returns the frequency response calculation of a linear system \verb!n/d! at frequencies. 

  The frequencies are given by the matrix \verb!frq! or computed from
  the bounds \verb!fmin! and \verb!fmax! (in Hz) and \verb!step! giving
  the ( logarithmic ) discretization step which is automatically adapted if
  (\verb!step='auto'!).
    
  \begin{description}
  \item[dom='c']: \verb!repf(k)=(n/d)(  2*%i*%pi*frq(k))! 
  \item[dom='d']]: \verb!repf(k)=(n/d)( 2*%i*%pi*dt*frq(k))! with \verb!dt=1! 
  \item[dom=x]]: \verb!repf(k)=(n/d)( 2*%i*%pi*dt*frq(k))! with \verb!dt=x! 
  \end{description}

  The vector \verb!frq! is splitted into regular parts with the \verb!split! vector i.e 
  \verb!frq(splitf(k):splitf(k+1)-1)! has no critical frequency. In other words, 
  the system \verb!n/d! has a pole in the range \verb![frq(splitf(k)),frq(splitf(k)+1)]! and 
  no poles outside.
\end{mandescription}
%--example 
% \begin{examples}
%   \begin{mintednsp}{nsp}
%   \end{mintednsp}
% \end{examples}
%-- see also
\begin{manseealso}
  \manlink{bode}{bode}
%  \manlink{freq}{freq} 
%  \manlink{calfrq}{calfrq} 
  \manlink{horner}{horner} 
  \manlink{nyquist}{nyquist} 
%  \manlink{dbphi}{dbphi}  
\end{manseealso}
%-- Author
\begin{authors}
  Serge Steer
\end{authors}
